\documentclass[electronics.tex]{subfiles}
\begin{document}
\chapter{Circuit analysis}
\tags{}
\label{ch:circuitanalysis}

\section{Sign convention}
\tags{}

We use the \keyword{passive sign convention} of electrical engineering, defined below and illustrated in \autoref{fig:sign_convetion_1}.
\tags{}

\begin{Definition}{passive sign convention}{passive_sign_convention}
	Power flowing \emph{in} to a component is considered to be \emph{positive} and power flowing \emph{out} of a component is considered \emph{negative}.
\end{Definition}

\begin{figure}[b]
	\centering
	\includegraphics[width=.7\linewidth]{figures/sign_convention_1.pdf}
	\caption{passive sign convention in terms of power $\mathcal{P}$.}
	\label{fig:sign_convetion_1}
\end{figure}

Because power $\mathcal{P} = v i$, this implies the current and voltage signs are prescribed by the convention.
For \keyword[passive element]{passive elements}, the electrical potential must drop in the direction of positive current flow.
This means the assumed direction of voltage drop across a passive element must be the same as that of the current flow.
For \keyword[active element]{active elements}, which supply power to the circuit, the converse is true:
the voltage drop and current flow must be in opposite directions.
\autoref{fig:sign_convetion_2} illustrates the possible configurations.
\tags{V, I, S}

\begin{figure}[b]
	\centering
	\includegraphics[width=1\linewidth]{figures/sign_convention_2.pdf}
	\caption{passive sign convention in terms of voltage $v$ and current $i$.}
	\label{fig:sign_convetion_2}
\end{figure}

When analyzing a circuit, for each passive element, draw an arrow beside it pointing in the direction of assumed current flow and voltage drop.
Try it out on \autoref{fig:sign_convetion_3}.
\tags{I, V, S}

\begin{figure}[t]
	\centering
	\includegraphics[width=.7\linewidth]{figures/sign_convention_3.pdf}
	\caption{an illustration of the passive sign convention on a circuit.}
	\label{fig:sign_convetion_3}
\end{figure}

The purpose of a sign convention is to help us \keyword[interpretation]{interpret} the signs of our results.
For instance, if, at a given instant, a capacitor has voltage $v_C = 3$ V and current $i_C = -2$ A, we compute $\mathcal{P}_C = -6$ W and we know $6$ W of power is flowing \emph{from} the capacitor into the circuit.
\tags{C, V, I, S, A}

For passive elements, there is no preferred direction of ``assumed'' voltage drop and current flow.
If a voltage or current value discovered by performing a circuit analysis is positive, this means the ``assumed'' and ``actual'' directions are the same. For a negative value, the directions are opposite.
\tags{V, I, S}

For active elements, \emph{we don't get to choose} the direction.
The physical situation prescribes it.
For instance, if a positive terminal of a battery is connected to a certain terminal in a circuit, I cannot simply say ``meh, I'm going to call that negative.''
It's positive whether you like it or not, Nancy.
\tags{}

\section{Methodology for analyzing circuits}
\tags{}
\label{lec:methodology_for_analyzing_circuits}

We have all the tools we need to do some pretty badass circuit analysis.
Later we'll learn a more systematic method for analyzing the dynamics of a circuit, but for now we can use broad strokes to get the idea.
It will work most of the time, but occasionally you may need to write some extra KCL or KVL equations or use a more advanced algebraic technique.
\tags{V, I, S}

Let $n$ be the number of passive circuit elements in a circuit, which gives $2n$ ($v$ and $i$ for each element) unknowns. The method is this.
\begin{enumerate}
	\item Draw a \emph{circuit diagram}.
	\item Label the circuit diagram with the \emph{sign assignment} by labeling each element with the ``assumed'' direction of current flow.
	\item Write the \emph{elemental equation} for each circuit element (e.g.\ Ohm's law).
	\item For every node not connected to a voltage source, write Kirchhoff's current law (KCL).
	\item For each loop not containing a current source, write Kirchhoff's voltage law (KVL).
	\item You probably have a linear system of $2 n$ algebraic and first-order, ordinary differential equations (and $2 n$ unknowns) to be solved simultaneously. 
	\begin{enumerate}
		\item Eliminate $n$ (half) of the unknowns by substitution into the elemental equations.
		\item Try substition or elimination to get down to only those variables with time derivatives and inputs. If this doesn't work, use a linear algebra technique.
		\item Solve the remaining set of first-order, linear ordinary differential equations. This can be done either directly or by turning it into a single higher-order differential equation and then solving.
	\end{enumerate}
\end{enumerate}

\examplemaybe{%
	RC circuit analysis with a constant source
}{%
	\begin{minipage}[c]{.5\linewidth}
    In the RC circuit shown, let $V_s(t) = 12$ V. If $v_C(t)|_{t=0} = 0$, what is $v_o(t)$ for $t\ge 0$?.
  \end{minipage}
  \hfill%
  \begin{minipage}[c]{.4\linewidth}
    \begin{circuitikz}[]
			\draw
				(0,0) to[voltage source, v=$V_s$] (0,2)
				to[R=$R$, i=$i_{R}$] (2,2)
				to[C=$C$, i=$i_{C}$] (2,0)
				-- (0,0);
			\draw
				(2,2) to[short, -o] (3,2)
				to[open, v^=$v_o$, -o] (3,0)
				to[short] (2,0);
		\end{circuitikz}
  \end{minipage}
}{%
	\begin{enumerate}
		\item The circuit diagram is given.
		\item The signs are given.
		\item The $n$ elemental equations are as follows.
			\begin{tabular}{l|r}
				$\begin{aligned}[t]
					C \\
					R
				\end{aligned}$ &
				$\begin{aligned}[t]
				\frac{d v_C}{d t} &= \frac{1}{C} i_C\\
				v_R &= i_R R
				\end{aligned}$ \\
			\end{tabular}
		\item There is one node not connected to the voltage source, for which the KCL equation is
		\begin{equation*}
			i_C = i_R.
		\end{equation*}
		\item There is one loop (and it doesn't have a current source in it), for which the KVL equation is
		\begin{equation*}
			v_R + v_C - V_s = 0.
		\end{equation*}
		\item Solve.
		\begin{enumerate}
			\item Eliminate $i_C$ and $v_R$ using KCL and KVL to yield the following.
				\begin{tabular}{l|r}
					$\begin{aligned}[t]
						C \\
						R
					\end{aligned}$ &
					$\begin{aligned}[t]
					\frac{d v_C}{d t} &= \frac{1}{C} i_R\\
					i_R &= \frac{1}{R}\left(V_s - v_C\right)
					\end{aligned}$ \\
				\end{tabular}
			\item Substituting the $R$ equation into the $C$ equation, we eliminate $i_R$ to obtain
				\begin{align*}
					\frac{d v_C}{d t} &= \frac{1}{RC} \left(V_s - v_C\right).
				\end{align*}
			\item Rearranging, 
				\begin{align*}
					R C \frac{d v_C}{d t} + v_C &=  V_s.
				\end{align*}
				Let's solve the ODE with initial condition $v_C(0) = 0$.
				\begin{enumerate}
					\item Homogeneous solution. The characterestic equation is
					\begin{align*}
						R C \lambda + 1 = 0 \quad \Rightarrow \quad \lambda = -\frac{1}{R C}.
					\end{align*}
					Therefore, $v_{Ch}(t) = k_1 e^{-t/(RC)}$.
					\item Particular solution. The form of the input $V_s(t)$ invites us to assume $v_{Cp}(t) = k_2$. Substituting into the ODE,
					\begin{align*}
						R C (0) + k_2 = 12\text{ V} \quad \Rightarrow \quad k_2 = 12\text{ V}.
					\end{align*}
					\item Total solution: $v_C(t) = v_{Ch}(t) + v_{Cp}(t)$.
					\item Specific solution:
					\begin{align*}
						v_C(0) &= 0 \quad \Rightarrow \quad
						k_1 + k_2 = 0 \quad \Rightarrow \quad
						k_1 = -k_2 = -12\text {V}.
					\end{align*}
					Since, $v_o(t) = v_C(t)$,
					\begin{align*}
						v_o(t) = 12 (1 - e^{-t/(R C)}).
					\end{align*}
					Should probably sketch this.
				\end{enumerate}
		\end{enumerate}
	\end{enumerate}
}{%
ex:rc_circuit_analysis_01%
}

\section{A sinusoidal input example}
\tags{}

Notice that we have yet to talk about \keyword[ac circuit analysis]{alternating current (ac) circuit analysis} or \keyword[dc circuit analysis]{direct current (dc) circuit analysis}.
In fact, these ambiguous terms can mean a few different things.
Approximately, an ac circuit analysis is one for which the input is sinusoidal and a dc circuit analysis is one for which the input is a constant.
This ignores \keyword{transient response} (early response when the initial-condition response dominates) versus \keyword{steady-state response} (later response when the initial-condition response has decayed) considerations.
We'll consider this more in \autoref{lec:transient_steady}.
\tags{}

We have remained general enough to be able to handle sinusoidal and constant sources in both transient and steady-state response.
\autoref{ex:rc_circuit_analysis_01} features a circuit with a constant voltage source and a capacitor.
Now we consider circuit with a sinusoidal current source and an inductor because why change only one thing when you could change more?
\tags{V, C, L, S, A, T}

\examplemaybe{%
	RL circuit analysis with a sinusoidal source
}{%
	\begin{minipage}[c]{.5\linewidth}
    Given the RL circuit shown, current input $I_s(t) = A \sin \omega t$, and initial condition $i_L(t)|_{t=0} = i_0$, what are $i_L(t)$ and $v_L(t)$ for $t\ge 0$?.
  \end{minipage}
  \hfill%
  \begin{minipage}[c]{.4\linewidth}
    \begin{circuitikz}[]
			\draw
				(0,0) to[current source, i=$I_s$] (0,2)
				-- (3,2)
				to[L=$L$, i=$i_{L}$] (3,0)
				-- (0,0);
			\draw
				(1.5,2) to[R=$R$, i=$i_{R}$] (1.5,0)
				node[ground]{};
			% \draw
			% 	(3,2) to[short, -o] (4,2)
			% 	to[open, v^=$v_o$, -o] (4,0)
			% 	to[short] (3,0);
		\end{circuitikz}
  \end{minipage}
}{%
	\begin{enumerate}
		\item The circuit diagram is given.
		\item The signs are given.
		\item The $n$ elemental equations are as follows.
			\begin{tabular}{l|r}
				$\begin{aligned}[t]
					L \\
					R
				\end{aligned}$ &
				$\begin{aligned}[t]
				\frac{d i_L}{d t} &= \frac{1}{L} v_L\\
				v_R &= i_R R
				\end{aligned}$ \\
			\end{tabular}
		\item There are actually two nodes not connected to the voltage source, but they give the same KCL equation
		\begin{equation*}
			i_R = I_s - i_L.
		\end{equation*}
		\item There is one loop that doesn't have a current source in it, for which the KVL equation is
		\begin{equation*}
			v_L = v_R.
		\end{equation*}
		\item Solve.
		\begin{enumerate}
			\item Eliminate $v_L$ and $i_R$ using KCL and KVL to yield the following.
				\begin{tabular}{l|r}
					$\begin{aligned}[t]
						L \\
						R
					\end{aligned}$ &
					$\begin{aligned}[t]
					\frac{d i_L}{d t} &= \frac{1}{L} v_R\\
					v_R &= (I_s - i_L) R
					\end{aligned}$ \\
				\end{tabular}
			\item Substituting the $R$ equation into the $L$ equation, we eliminate $v_R$ to obtain
				\begin{align*}
					\frac{d i_L}{d t} &= \frac{R}{L} \left(I_s - i_L\right).
				\end{align*}
			\item Rearranging and letting $\tau = L/R$, 
				\begin{align*}
					\tau \cdot \frac{d i_L}{d t} + i_L &=  I_s.
				\end{align*}
				Let's solve the ODE with initial condition $i_L(0) = i_0$.
				\begin{enumerate}
					\item Homogeneous solution. The characterestic equation is
					\begin{align*}
						\tau \lambda + 1 = 0 \quad \Rightarrow \quad \lambda = -\frac{1}{\tau}.
					\end{align*}
					Therefore, $i_{Lh}(t) = \kappa_1 e^{-t/\tau}$.
					\item Particular solution. The form of the input $I_s(t)$ invites us to assume $i_{Lp}(t) = k_1 \sin\omega t + k_2 \cos\omega t$. Substituting into the ODE,
					\begin{align*}
						\tau\omega (k_1 \cos\omega t - k_2 \sin\omega t)\ + \\ 
						(k_1 \sin\omega t + k_2 \cos\omega t) =&\ A \sin\omega t \quad \Rightarrow
					\end{align*}
					\begin{equation*}
						\begin{aligned}[c]
							&\tau\omega k_1 + k_2 = 0 \\
							k_2 &= -\tau\omega k_1  \\
							k_1 &= \frac{A}{(\tau\omega)^2+1}
						\end{aligned}
						\quad
						\begin{aligned}[c]
							\text{and} \\
							\text{and} \\
							\text{and}\vphantom{\frac{1}{1}}
						\end{aligned}
						\quad
						\begin{aligned}[c]
							-\tau\omega k_2 + k_1 = A& \quad \Rightarrow\\
							(\tau\omega)^2 k_1 + k_1 = A& \quad \Rightarrow\\
							k_2 = -\frac{A\tau\omega}{(\tau\omega)^2+1}&
						\end{aligned}
					\end{equation*}
					Success! So we have
					\begin{align*}
						i_{Lp}(t) = \frac{A}{(\tau\omega)^2+1} \left( 
							\sin\omega t - \tau\omega \cos\omega t
						\right).
					\end{align*}
					\item Total solution: 
					\begin{align*}
						i_L(t) &= i_{Lh}(t) + i_{Lp}(t) \\
						&= \kappa_1 e^{-t/\tau} +
						\frac{A}{(\tau\omega)^2+1} \left( 
							\sin\omega t - \tau\omega \cos\omega t
						\right).
					\end{align*}
					\item Specific solution:
					\begin{align*}
						i_L(0) &= i_0 \quad \Rightarrow \\
						\kappa_1 - 
						\frac{A \tau\omega}{(\tau\omega)^2+1} &= i_0 \quad \Rightarrow \\
						\kappa_1 &= i_0 +
						\frac{A \tau\omega}{(\tau\omega)^2+1}.
					\end{align*}
					Therefore,
					\begin{align*}
						i_L(t) =
						&\left(
							i_0 +
							\frac{A \tau\omega}{(\tau\omega)^2+1}
						\right)
						e^{-t/\tau} + \\
						&\frac{A}{(\tau\omega)^2+1} 
						\left( 
							\sin\omega t - \tau\omega \cos\omega t
						\right).
					\end{align*}
					Now, a better way to write this is as a single sinusoid. The two-to-one formulas of \autoref{eq:two_to_one} can be applied to obtain
					\begin{align*}
						i_L(t) =
						&\left(
							i_0 +
							\frac{A \tau\omega}{(\tau\omega)^2+1}
						\right)
						e^{-t/\tau} + \\
						&\frac{A}{\sqrt{(\tau\omega)^2+1}}  
							\sin(\omega t - \arctan \tau\omega).
					\end{align*}
					We can see clearly that there is a sinusoidal component and a decaying exponential component.
				\end{enumerate}
		\end{enumerate}
	\item Finally, we can find $v_L(t)$ from the inductor elemental equation:
		\begin{align*}
			v_L(t) &=
			L \frac{d i_L}{d t} \\
			&= -\frac{L}{\tau}
			\left(
				i_0 +
				\frac{A \tau\omega}{(\tau\omega)^2+1}
			\right)
			e^{-t/\tau} + \\
			&\frac{AL\omega}{\sqrt{(\tau\omega)^2+1}}  
				\cos(\omega t - \arctan \tau\omega).
		\end{align*}
	\item See \autoref{lec:transient_steady} for plots.
	\end{enumerate}
}{%
ex:rl_circuit_analysis_01%
}

\section{Transient and steady-state response}
\tags{}
\label{lec:transient_steady}.

\input{source/transient_steady}


\begin{exercises}
\input{ch02_exercises}
\end{exercises}

\end{document}
