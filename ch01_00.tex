\documentclass[electronics.tex]{subfiles}

%% tags tagging
\usepackage{lstdoc}

\makeatletter
\def\tags{R, L, V, I, C,}
%% sorts the tags for later on
\def\tags@sorted{\lst@BubbleSort\tags}
\tags@sorted
%% we automatically create macros for each tag
 \def\macrofy@#1\@nil{%
      \expandafter\def\csname#1\endcsname{}
  }
\def\addtags#1{\g@addto@macro\tags{#1,}
  \tags@sorted
  \macrofy@@
}
%% Make macros to hold the lists for each tag
%% 
\def\macrofy@@{\@for\next:=\tags\do{%
    \expandafter\macrofy@\next\@nil
}}
\macrofy@@

\def\addtotag#1\@nil#2{%
  \expandafter\ifx\csname#2\endcsname\@empty
        \expandafter\g@addto@macro\csname#2\endcsname{#1}
    \else
        \expandafter\g@addto@macro\csname#2\endcsname{,#1}
    \fi
}
\def\tags#1{%
  \expandafter\addtotag\the\c@section\@nil{#1}
}
\def\thetags{%
%% we now print the tags or save them to a file
  \section*{The Tags}  
  \@for\next:=\tags\do{%
     \ifx\next\@empty\else\next: See Section(s) \@nameuse{\next}\par\fi
  }
}

\newcommand{\keyword}[2][]{%
	\ifthenelse{\isempty{#1}}{%
		\mymarginpar{%
			\footnotesize\sffamily\bfseries\textcolor{mygreen}{#2}%
		}%
		\emph{#2}%
	}{%
		\mymarginpar{%
			\footnotesize\sffamily\bfseries\textcolor{mygreen}{#1}%
		}%
		\emph{#2}%
	}%
}%


\begin{document}
\chapter{Fundamentals}

Read \cite{Horowitz2015}.

\section{Voltage, current, resistance, and all that}
\tags{}

Two quantities will be of special importance in analyzing and designing electronic systems: \keyword{voltage} and \keyword{current}.
The relationship between them defines a third important quantity: \keyword{resistance} (more generally, \emph{impedance}).
\tags{V, I}

Momentarily, we will define each of these, but we start with the fundamental quantity in electronics.
\tags{}

\begin{Definition}{electric charge}{charge}
  Electric charge (or simply charge) is a  property of matter that describes the attractive or repulsive force acting on the matter in an electric field. At the microscopic level, charge is quantized into charges of subatomic particles such as protons and electrons, which have opposite charges $e$ and $-e$, where $e$ is the \href{https://en.wikipedia.org/wiki/Elementary_charge}{elementary charge}.
\end{Definition}

Charge has derived SI unit \keyword{coulomb} with symbol C.
It is considered to be a \keyword{conserved quantity}.
\tags{}

\subsection{Voltage}
\tags{V}

\begin{Definition}{voltage}{voltage}
  Voltage is the difference in electrical potential energy of a unit of charge moved between two locations in an electrical field.
\end{Definition}

Voltage is typically given the variable $v$ and has derived SI unit \keyword{volt} with symbol V.
\tags{}

Voltage is always defined by referring to \emph{two} locations.
Sometimes one of these locations is implicitly \keyword{ground}---an arbitrarily-defined reference (datum) voltage considered to have zero electrical potential energy---such that we can talk about the voltage ``at this'' or ``at that'' location by implicit reference to ground.
It is good form to describe the voltage as being ``between'' two locations or ``across'' an element.
\tags{}

\subsection{Current}
\tags{I}

\begin{Definition}{current}{current}
  Current is a flow of charge.
\end{Definition}

Current is typically denoted $i$ and has derived SI unit \keyword{ampere} with symbol~A.
\tags{}

We typically \emph{generate} voltage by doing work on charges.
Conversely, we \emph{get} currents by placing voltage across matter through which current can flow.
This implies that voltage causes current.
Causality here is quite complex, but I will posit the following proposition.
We typically observe current when applying
voltage, so from a phenomenological point-of-view, it is natural to consider voltage causal of current.\footnote{Note that subtlety emerges not only when considering fields, small distances, and short durations---it also emerges when we consider certain circuit elements that are exhibit behavior related to the time rate of change of voltage or current.}
\tags{V}

\subsection{Circuits}
\tags{}

Electric \keyword[circuit]{circuits} are dynamic electrical systems in which charge accumulates in and flows through elements.
Circuit elements are connected via metallic conductors called \keyword[wire]{wires}, which ideally have the same voltage (relative to, say, ground) everywhere.
\tags{V}

\subsection{Circuit topology}
\tags{}

A circuit has a few basic topological features.
\tags{}

A circuit \keyword{node} is a continuous region of a circuit that has the same voltage everywhere.
A node is an idealized concept that is approximate in most instantiations.
\tags{}

A circuit \keyword{element} is a region of a circuit considered to have properties distinct from the surrounding circuit.
Examples of elements are resistors, capacitors, inductors, and sources.
\tags{ex, R, C, L}

A circuit element has \keyword{terminals} through which it connects to a circuit.
\tags{}

Circuit elements in \keyword{parallel} are those that have two terminals, each of which is shared by another element's two terminals.
\tags{}

Circuit elements in \keyword{series} are those that have two terminals, only one of which is shared between them and this one cannot be shared with any other element.
\tags{}

\subsection{Element types}

The following are common types of circuit element.
\begin{itemize}
  \item \keyword[energy storage element]{Energy storage elements} store energy in electric (capacitors) or magnetic (inductors) fields.
  \item \keyword[energy dissipative element]{Energy dissipative elements} dissipate energy from a circuit, typically as heat, such as in a resistor.
  \item \keyword[energy source element]{Energy source elements} provide external energy to the circuit (e.g.\ batteries).
  \item \keyword[energy transducing element]{Energy transducing elements} convert electronic energy to another form (e.g.\ motors convert electric to mechanical energy.)
\end{itemize}

\subsection{Power}
\tags{}

Power is the time rate of change of energy.
Let us now define electric power.
\tags{}

\begin{Definition}{power}{power}%
  The instantaneous electric power $\mathcal{P}$ into a circuit element is defined as the product of the voltage $v$ across and the current $i$ through it at a given time $t$:
  \begin{align}
    \mathcal{P}(t) = v(t) i(t).
  \end{align}
\end{Definition}

Power typically goes into:
\tags{}

\begin{itemize}
  \item heat (usually),
  \item mechanical work (motors),
  \item radiated energy (lamps, transmitters), or
  \item stored energy (batteries, capacitors).
\end{itemize}

\begin{infobox}[terminological note]
  ``[D]on't call current `amperage'; that's strictly bush-league. The same caution will apply to the term `ohmage' ....''\\
  {\raggedleft---Horowitz \& Hill, The Art of Electronics\par}
\end{infobox}

\subsection{Kirchhoff's laws}
\tags{}

Gustav Kirchhoff formulated two laws fundamental to circuit analysis.
\tags{}

Kirchhoff's current law (KCL) depends on the fact that charge is a conserved quantity. Therefore, the charge flowing \emph{in} a node is equal to that flowing \emph{out}, which implies KCL.
\tags{}

\begin{Definition}{Kirchhoff's current law}{KCL}
  The current \emph{in} a node is equal to the current \emph{out}.
\end{Definition}

KCL implies that the sum of the current into a node must be zero.
Assume, for instance, that $k$ wires with currents $i_j$ connect to form a node.
Kirchhoff's current law states that
\tags{I}

\begin{align}
  \sum_{j=1}^k i_j = 0.
\end{align}

It can be discovered empirically that elements connected in parallel have the same voltage across them.
This doesn't mean they share the same current, but it does imply Kirchhoff's voltage law (KVL).
\tags{V}

\begin{Definition}{Kirchhoff's voltage law}{KVL}
  The sum of the voltage drops around any closed loop is zero.\footnote{A loop is a series of elements that begins and ends at the same node.}
\end{Definition}

KVL implies that the voltage drops across elements that form a loop must be zero.
Assume, for instance, that $k$ elements with voltage drops $v_j$ form a loop.
KVL states that
\tags{V}

\begin{align}
  \sum_{j=1}^k v_j = 0.
\end{align}

\subsection{Ohm's law}
\tags{}

Much of electronics is about the relationship between a voltage and a corresponding current.
Applying a voltage to a material typically induces a current through it.
The functional relationship between $v$ and $i$ is of the utmost importance to the analysis and design of circuits.
\tags{V, I}

The simplest relationship is known as Ohm's law, for which we will first need the concept of resistance.
\tags{R}

\begin{Definition}{resistance}{resistance}
  Let a circuit element have voltage $v$ and current $i$. The resistance $R$ is defined as the ratio
  \begin{align}
    R = v/i
  \end{align}
\end{Definition}

Now we are ready to define Ohm's law.
\tags{}

\begin{Definition}{Ohm's law}{ohm}
  Some materials such as conductors in certain environments exhibit approximately constant resistance.
\end{Definition}

This is pretty weak.
However, it's still quite useful, as we'll see.
With it we can assume, for certain elements and situations, that the resistance of the element is a static property and that the voltage and current are proportional.
We call such elements \keyword[resistor]{resistors}.
\tags{V, I, R}

\subsection{Combining resistance}
\tags{R}

Resistors can be connected together in different topologies to form composite elements that exhibit ``equivalent'' resistances of their own.
\tags{}

$K$ resistors with resistances $R_j$ connected in \emph{series} have equivalent resistance $R_e$ given by the expression
\tags{}

\begin{align}
  R_e = \sum_{j=1}^K R_j.
\end{align}

$K$ resistors with resistances $R_j$ connected in \emph{parallel} have equivalent resistance $R_e$ given by the expression
\tags{}

\begin{align}
  R_e = 1/\sum_{j=1}^K 1/R_j.
\end{align}

In the special case of two resistors with resistances $R_1$ and $R_2$,
\maybeeq{%
\begin{align}
  R_e
  &= \frac{1}{\frac{1}{R_1}+\frac{1}{R_2}} \nonumber \\
  &= \frac{R_1 R_2}{R_1 + R_2}.
\end{align}
}

\examplemaybe{
  understanding a circuit
  }{
  
  Answer the questions below about the circuit shown. Voltage across and current through a circuit element $x$ are denoted $v_x$ and $i_x$. Signs are defined on the diagram.
  \tags{V, I}
  
  \begin{enumerate}
    \item What does it mean if we refer to the voltage at node \emph{a}?
    \item What is the current $i_{R_2}$ through $R_2$ at a given time $t$ in terms of the power it is dissipating $\mathcal{P}_{R_2}$ and the voltage across it $v_{R_2}$?
    \item If $V_s(t) = 5$ V and $v_{R_1} = 3$ V, what is $v_{R_2}$?
    \item What is the equivalent resistance of the resistors $R_1$ and $R_2$ combined as in the circuit?
    \item If $v_{R_1} = 3$ V and $R_1 = 100\ \Omega$, what is $i_{R_2}$?
  \end{enumerate}
  \begin{centering}
    \includegraphics{figures/ex_voltage_divider_01.pdf}
  \end{centering}
  }{
  \begin{enumerate}
    \item We mean the voltage from $a$ to ground.
    \item From the definition of the instantaneous power,
    \begin{align*}
      i_{R_2}(t) = \mathcal{P}_{R_2}(t)/v_{R_2}(t).
    \end{align*}
    \item From KVL,
    \begin{align*}
      v_{R_2} &= V_s - v_{R_1} \\
      &= 5 - 3 \\
      &= 2\ \text{V}.
    \end{align*}
    \item In the circuit, $R_1$ and $R_2$ are in \emph{series}, so they can be combined as
    \begin{align*}
      R_e = R_1 + R_2.
    \end{align*}
    \item We can write down Ohm's law for $R_1$:
    \begin{align*}
      v_{R_1} &= i_{R_1} R_1.
    \end{align*}
    So we know
    \begin{align*}
      i_{R_1} &= v_{R_1}/R_1 \\
      &= 30\ \text{mA}.
    \end{align*}
    From KCL, $i_{R_2} = i_{R_1}$. Therefore,
    \begin{align*}
      i_{R_2} = 30\ \text{mA}.
    \end{align*}
  \end{enumerate}
}{%
ex:voltage_divider_01%
}

\section{Voltage dividers}
\tags{}
\label{lec:voltage_dividers}

In \autoref{ch:circuitanalysis} we'll learn about how to approach circuit analysis in a systematic way.
For now, we'll limp along unsystematically with our toolbelt of concepts and equations in order to introduce some more circuit elements, concepts, and theorems.
But we can't resist just a bit of circuit analysis now.
\tags{}

The \keyword{voltage divider} is a ubiquitous and useful circuit.
In a sense, it's less of a circuit and more of concept.
For resistors, that concept can be stated as the following.
\tags{R, V}
\begin{quote}
  The voltage across resistors in series is divided among the resistors.
\end{quote}

An immediately useful result is that we can ``divide voltage'' into any smaller voltage we like by putting in a couple resistors.
\tags{V, R}
\begin{figure}[b]
  \centering
  \includegraphics[width=.6\linewidth]{figures/voltage_divider_demo.pdf}
  \caption{a simple voltage divider circuit.}
  \label{fig:voltage_divider_demo}
\end{figure}

In order to show \emph{how} the voltage divider ``divides up'' the voltage, we must do some basic circuit analysis.
Consider the circuit in \autoref{fig:voltage_divider_demo}.
The input voltage $v_\text{in}$ is divided into $v_{R_1}$ and $v_{R_2} = v_\text{out}$.
We want to know $v_\text{out}$ as a function of $v_\text{in}$ and parameters $R_1$ and $R_2$.
Let's write down the equations we know from the laws of Kirchhoff and Ohm:
\tags{V, R}
\maybeeq{%
\begin{align*}
  v_{R_1} &= i_{R_1} R_1, \\
  v_{R_2} &= i_{R_2} R_2, \\
  v_\text{in} &= v_{R_1} + v_{R_2},\text{ and} \\
  i_{R_1} &= i_{R_2}.
\end{align*}
}%
We've already established that $v_\text{out} = v_{R_2}$, so we can solve for $v_{R_2}$ in ($*$).
We want to eliminate the three ``unknown'' variables $v_{R_1}$, $i_{R_1}$, and $i_{R_2}$, so it is good that we have four equations.\footnote{Alternatively, we could solve for all four unknown variables with our four equations.}
We begin with ($*$b) and proceed by substitution of the others of ($*$):

\maybeeq{%
\begin{align*}
  v_{R_2} &= i_{R_2} R_2 \\
  &= i_{R_1} R_2 \\
  &= \frac{R_2}{R_1} v_{R_1} \\
  &= \frac{R_2}{R_1} \left( v_\text{in} - v_{R_2} \right) \ \Rightarrow \\
  v_{R_2} + \frac{R_2}{R_1} v_{R_2} &= \frac{R_2}{R_1} v_\text{in} \ \Rightarrow \\
  v_{R_2} &= \frac{R_2/R_1}{1+R_2/R_1} v_\text{in} \ \Rightarrow \\
  &= \frac{R_2}{R_1+R_2} v_\text{in}.
\end{align*}
}

Nice! So we can now write the input-output relationship for a two-resistor voltage divider.
\maybeeqn{two-resistor voltage divider}{eq:voltage_divider_two}{%
\begin{align*}
  v_\text{out} &= \frac{R_2}{R_1+R_2} v_\text{in}.
\end{align*}
}

So the voltage divider had the effect of dividing the input voltage into a fraction governed by the relationship between the relative resistances of the two resistors.
This fraction takes values in the interval $[0,1]$.
Now, whenever we see the voltage divider circuit, we can just remember this easy formula!
\tags{V, R}

Similarly, for $n$ resistors in series, it can be shown that the voltage divider relationship is as follows.
\tags{R, V}
\maybeeqn{general voltage divider}{eq:voltage_divider}{%
The output voltage $v_\text{out}$ across resistor $R_k$ in a series of $n$ resistors with input $v_\text{in}$ is
\begin{align*}
  v_\text{out} &= \frac{R_k}{R_1+R_2+\cdots+R_k+\cdots+R_n} v_\text{in}.
\end{align*}
}

\section{Sources}
\tags{}

Sources (a.k.a.\ supplies) supply power to a circuit. There are two primary types: \emph{voltage sources} and \emph{current sources}.
\tags{V, I}

\subsection{Ideal voltage sources}
\tags{V}

An ideal voltage source provides exactly the voltage a user specifies, independent of the circuit to which it is connected.
All it must do in order to achieve this is to supply whatever current necessary.
Let's unpack this with a simple example.
\tags{}

\examplemaybe{
  limitations of a voltage source
  }{%
  \begin{minipage}[c]{.6\linewidth}
    In the circuit shown, determine how much current and power the ideal voltage source $V_s$ must provide in order to maintain voltage if $R\rightarrow \infty$ and if $R\rightarrow 0$.
  \end{minipage}
  \hfill%
  \begin{minipage}[c]{.3\linewidth}
    \includegraphics{figures/ex_voltage_source.pdf}
  \end{minipage}
  }{
  Since the voltage across the resistor is known to be equal to $V_s$, Ohm's law tells us that
  \begin{align*}
    i_R = V_s/R.
  \end{align*}
  Of course, power dissipated by the resistor is
  \begin{align*}
    \mathcal{P}_R &= i_R v_R \\
    &= V_s^2/R.
  \end{align*}
  Taking the ``open circuit'' limit,
  \begin{align*}
    i_{R\rightarrow \infty} &= \lim_{R\rightarrow \infty} V_s/R \\
    &= 0\text{ A} \\
    \mathcal{P}_{R\rightarrow \infty} &= \lim_{R\rightarrow \infty} V_s^2/R \\
    &= 0\text{ W}.
  \end{align*}
  Taking the ``short circuit'' limit,
  \begin{align*}
    i_{R\rightarrow 0} &= \lim_{R\rightarrow 0} V_s/R \\
    &\rightarrow \infty\text{ A} \\
    \mathcal{P}_{R\rightarrow 0} &= \lim_{R\rightarrow 0} V_s^2/R \\
    &\rightarrow \infty\text{ W}.
  \end{align*}
}{%
ex:voltage_source%
}

\subsection{Ideal current sources}
\tags{I}

An ideal current source provides exactly the current a user specifies, independent of the circuit to which it is connected.
All it must do in order to achieve this is to supply whatever voltage necessary.
Let's unpack this with a simple example.
\tags{}

\examplemaybe{
  limitations of a current source
  }{%
  \begin{minipage}[c]{.6\linewidth}
    In the circuit shown, determine how much voltage and power the ideal current source $I_s$ must provide in order to maintain voltage if $R\rightarrow 0$ and if $R\rightarrow \infty$.
  \end{minipage}
  \hfill%
  \begin{minipage}[c]{.3\linewidth}
    \includegraphics{figures/ex_current_source.pdf}
  \end{minipage}
  }{
  Since the current through the resistor is known to be equal to $I_s$, Ohm's law tells us that
  \begin{align*}
    v_R = I_s R.
  \end{align*}
  Of course, power dissipated by the resistor is
  \begin{align*}
    \mathcal{P}_R &= i_R v_R \\
    &= I_s^2 R.
  \end{align*}
  Taking the ``short circuit'' limit,
  \begin{align*}
    v_{R\rightarrow 0} &= \lim_{R\rightarrow 0} I_s R \\
    &\rightarrow 0\text{ A} \\
    \mathcal{P}_{R\rightarrow 0} &= \lim_{R\rightarrow 0} I_s^2 R \\
    &\rightarrow 0\text{ W}.
  \end{align*}
  Taking the ``open circuit'' limit,
  \begin{align*}
    v_{R\rightarrow \infty} &= \lim_{R\rightarrow \infty} I_s R \\
    &= \infty\text{ A} \\
    \mathcal{P}_{R\rightarrow \infty} &= \lim_{R\rightarrow \infty} I_s^2 R \\
    &= \infty\text{ W}.
  \end{align*}
}{%
ex:current_source%
}

\subsection{Modeling real sources}
\tags{}

No real source can produce infinite power.
Some have feedback that controls the output within some finite power range.
These types of sources can be approximated as ideal when operating within their specifications.
Many voltage sources (e.g.\ batteries) do not have internal feedback controlling the voltage.
When these sources are ``loaded'' (delivering power) they cannot maintain their nominal output, be that voltage or current.
We model these types of sources as ideal sources in series or parallel with a resistor, as illustrated in \autoref{fig:real_sources}.
\tags{V, I}

\begin{figure}[b]%
  \centering
  \subbottom[real voltage source model\label{fig:voltage_source_real}]{
    \includegraphics{figures/voltage_source_real.pdf}
  }%
  \qquad
  \subbottom[real current source model.\label{fig:current_source_real}]{
    \includegraphics{figures/current_source_real.pdf}
  }
  \caption{Models for power-limited ``real'' sources.}%
  \label{fig:real_sources}%
\end{figure}

Most manufacturers specify the nominal resistance of a source as the ``output resistance.''
A typical value is $50\ \Omega$.
\tags{R}
\section{Thevenin's and Norton's theorems}
\tags{}

Th\'evenin's and Norton's theorems yield ways to simplify our models of circuits.
\tags{}

\subsection{Th\'evenin's theorem}
\tags{}

The following remarkable theorem has been proven.
\tags{}

\begin{Theorem}{Th\'evenin's theorem}{th:thevenin}
  Given a linear network of voltage sources, current sources, and resistors, the behavior at the network's output terminals can be reproduced exactly by a single \emph{voltage source $V_e$ in series with a resistor $R_e$}.
\end{Theorem}

The equivalent circuit has two quantities to determine: $V_e$ and $R_e$.
\tags{V, R}

\subsubsection{Determining $R_e$}
\tags{R}

The \keyword{equivalent resistance $R_e$} of a circuit is the resistance between the output terminals with all inputs set to zero.
Setting a voltage source to zero means the voltage on both its terminals are equal, which is equivalent to treating it as a short or wire.
Setting a current source to zero means the current through it is zero, which is equivalent to treating it as an open circuit.
\tags{V, I}

\subsubsection{Determining $V_e$}
\tags{V}

The \keyword{equivalent voltage source $V_e$} is the voltage at the output terminals of the circuit when they are left open (disconnected from a load).
Determining this value typically requires some circuit analysis with the laws of Ohm and Kirchhoff.
\tags{}

\subsection{Norton's theorem}
\tags{}

Similarly, the following remarkable theorem has been proven.
\tags{}

\begin{Theorem}{Norton's theorem}{th:norton}
  Given a linear network of voltage sources, current sources, and resistors, the behavior at the network's output terminals can be reproduced exactly by a single \emph{current source $I_e$ in parallel with a resistor $R_e$}.
\end{Theorem}

The equivalent circuit has two quantities to determine: $I_e$ and $R_e$.
The equivalent resistance $R_e$ is identical to that of Th\'evenin's theorem, which leaves the equivalent current source $I_e$ to be determined.
\tags{V, I, R}

\subsubsection{Determining $I_e$}
\tags{I}

The \keyword{equivalent current source $I_e$} is the current through the output terminals of the circuit when they are shorted (connected by a wire).
Determining this value typically requires some circuit analysis with the laws of Ohm and Kirchhoff.
\tags{}

\subsection{Converting between Th\'evenin and Norton equivalents}
\tags{}

There is an equivalence between the two equivalent circuit models that allows one to convert from one to another with ease.
The equivalent resistance $R_e$ is identical in each and provides the following equation for converting between the two representations:
\tags{R}

\maybeeqn{converting between Th\'evenin and Norton equivalents}{eq:converting_thevenin_norton}{
  \begin{align}
    V_e = R_e I_e.
  \end{align}
}

\examplemaybe{
  Th\'evenin and Norton equivalents
  }{%
  \begin{minipage}[c]{.4\linewidth}
    For the circuit shown, find a Th\'evenin and a Norton equivalent.
  \end{minipage}
  \hfill%
  \begin{minipage}[c]{.5\linewidth}
    \includegraphics[width=1\linewidth]{figures/ex_thevenin_norton.pdf}
  \end{minipage}
  }{
  \noindent
  \begin{minipage}[c]{.6\linewidth}
    The Th\'evenin equivalent is shown. Now to find $R_e$ and $V_e$. Setting $V_s = 0$, $R_1$ and $R_2$ are in parallel, combining to give
    \begin{align*}
      R_e = \frac{R_1 R_2}{R_1+R_2}.
    \end{align*}
  \end{minipage}
  \hfill%
  \begin{minipage}[c]{.3\linewidth}
    \includegraphics[width=1\linewidth]{figures/thevenin_equivalent.pdf}
  \end{minipage}
  \newline
  Now to find $v_\text{out}$. It's a voltage divider:
  \begin{align*}
    V_e = v_\text{out} = \frac{R_2}{R_1+R_2} V_s.
  \end{align*}
  \begin{minipage}[c]{.6\linewidth}
    The Norton equivalent is shown. We know $R_e$ from the Th\'evenin equivalent, which also yields
    \begin{align*}
      I_e = V_e/R_e.
    \end{align*}
  \end{minipage}
  \hfill%
  \begin{minipage}[c]{.3\linewidth}
    \includegraphics[width=1\linewidth]{figures/norton_equivalent.pdf}
  \end{minipage}
}{%
ex:equivalent_sources%
}

\section{Output and input resistance and circuit loading}
\tags{}

When considering a circuit from the perspective of two terminals---either as \emph{input} or \emph{output}---it is often characterized as having a Th\'evenin/Norton \keyword{equivalent resistance} and, if it is considered as an output, as having an equivalent (Th\'evenin or Norton) source.
\tags{R}

If the terminals are considered to be an \emph{output}, its \keyword{output resistance} is just the Th\'evenin/Norton equivalent resistance.
Other names for this output resistance are \emph{source} or \emph{internal resisistance}.\footnote{Sometimes, instead of \emph{resistance}, the term \emph{impedance} is substituded. In these situations, there is no difference in meaning.}
\autoref{fig:output_input_resistance} illustrates this model.
\tags{R}

\begin{figure}[b]
  \centering
  \includegraphics{figures/output_input_resistance.pdf}
  \caption{source with Th\'evenin equivalent source voltage $V_e$ and output/internal resistance $R_e$ and a load with input resistance $R_L$.}
  \label{fig:output_input_resistance}
\end{figure}

If the terminals are considered to be an \emph{input}, its \keyword{input resistance} is the is the Th\'evenin/Norton equivalent resistance of the circuit.
Another term for this input resistance is the \emph{load resistance}.
\tags{R}

\subsection{Loading the source}
\tags{}

\keyword[loading a source]{Loading a source} means to connect another circuit to it that draws power.
Let's explore what happens when we connect the load to the source for the circuit in \autoref{fig:output_input_resistance}.
\tags{}

Before connecting, the source output voltage is
\tags{V}
\maybeeq{%
  \begin{align*}
    v_\text{out}
    &= V_e - v_{R_e} \\
    &= V_e - \cancelto{0}{i}_{R_e} R_e \\
    &= V_e.
  \end{align*}
}
This is equivalent to connecting a load with an infinite resistance.
After connecting, we have a voltage divider, so
\maybeeq{%
  \begin{align*}
    v_\text{out}
    &= \frac{R_L}{R_e+R_L} V_e \\
    &= \frac{1}{1+R_e/R_L} V_e.
  \end{align*}
}
So, as $R_e/R_L\rightarrow 0$, $v_\text{out} \rightarrow V_e$.
Also, as $R_e/R_L\rightarrow\infty$, $v_\text{out}\rightarrow 0$.

So, relatively small output resistance and large input resistance yield a ``loaded'' voltage nearer nominal.
Some sources are labeled with nominal values assuming no load and others assuming a \keyword{matching load}\footnote{A matching load can be shown to have maximum power transfer.}---a load equal to the output impedance.
For this reason, it is best to measure the actual output of any source.
\tags{R, V}

\section{Capacitors}
\tags{C}
Capacitors have two terminal and are composed of two conductive surfaces separated by some distance.
One surface has charge $q$ and the other $-q$.
A capacitor stores energy in an \emph{electric field} between the surfaces.

Let a capacitor with voltage $v$ across it and charge $q$ be characterized by the parameter \keyword{capacitance} $C$, where the constitutive equation is
\tags{V}

\maybeeq{%
  \begin{align}
    q = C v.
  \end{align}
}
The capacitance has derived SI unit \keyword[farad (F)]{farad (\emph{F})}, where $\text{F} = \text{A}\cdot\text{s}/\text{V}$.
A farad is actually quite a lot of capacitance.
Most capacitors have capacitances best represented in $\mu\text{F}$, nF, and pF.
\tags{}

The time-derivative of this equation yields the $v$-$i$ relationship (what we call the ``elemental equation'') for capacitors.
\tags{}

\maybeeqn{capacitor elemental equation}{eq:elemental_capacitor}{%
  \begin{align*}
    \frac{d v}{d t} = \frac{1}{C}\, i
  \end{align*}
}

A time-derivative!
This is new.
Resistors have only algebraic $i$-$v$ relationships, so circuits with only sources and resistors can be described by \emph{algebraic} relationships.
The dynamics of circuits with capacitors are described with \emph{differential equations}.
\tags{R}

\begin{figure}[b]%
  \centering
  \subbottom[bipolar capacitor.\label{fig:bipolar_capacitor}]{
    \begin{circuitikz}[]
      \draw
        (0,0) to[C=$C$] (2,0);
      \draw
        (-1,0) to[open] (3,0);
    \end{circuitikz}
  }%
  \quad
  \subbottom[polarized capacitor\label{fig:polarized_capacitor}]{
    \begin{circuitikz}[]
      \draw
        (2,0) to[pC, l_=$C$, v>=$ $] (0,0);
      \draw
        (-1,0) to[open] (3,0);
    \end{circuitikz}
  }
  \caption{capacitor circuit diagram symbols.}%
  \label{fig:capacitors}%
\end{figure}

Capacitors allow us to build many new types of circuits: filtering, energy storage, resonant, blocking (blocks dc-component), and bypassing (draws ac-component to ground).
\tags{}

Capacitors come in a number of varieties, with those with the largest capacity (and least expensive) being \keyword[electrolytic capacitor]{electrolytic} and most common being \keyword[ceramic capacitor]{ceramic}. There are two functional varieties of capacitors: \keyword[bipolar capacitor]{bipolar} and \keyword[polarized capacitor]{polarized}, with circuit diagram symbols shown in \autoref{fig:capacitors}.
Polarized capacitors can have voltage drop across in only one direction, from \keyword{anode} ($+$) to \keyword{cathode} ($-$)---otherwise they are damaged or may \keyword[explosion]{explode}.
Electrolytic capacitors are polarized and ceramic capacitors are bipolar.
\tags{V, C}

So what if you need a high-capacitance bipolar capacitor?
Here's a trick: place identical high-capacity polarized capacitors \keyword{cathode-to-cathode}.
What results is effectively a bipolar capacitor with capacitance \emph{half} that of one of the polarized capacitors.
\tags{C}

\section{Inductors}
\tags{L}

\begin{wrapfigure}{R}{0.3\textwidth}
  \centering
  \begin{circuitikz}[]
    \draw
      (0,0) to[L=$L$] (2,0);
  \end{circuitikz}
  \caption{\label{fig:inductor} inductor circuit diagram symbol.}%
\end{wrapfigure}

A \keyword{pure inductor} is defined as an element in which \keyword{flux linkage $\lambda$}---the integral of the voltage---across the inductor is a monotonic function $\mathcal{F}$ of the current $i$; i.e.\ the pure constitutive equation is
\tags{V}
\begin{align}
  \lambda = \mathcal{F}(i).
\end{align}
An \keyword{ideal inductor} is such that this monotonic function is linear, with slope called the \keyword{inductance $L$}; i.e.\ the ideal constitutive equation is
\maybeeq{%
  \begin{align*}
    \lambda = L i
  \end{align*}
}

The units of inductance are the SI derived unit \keyword[henry (H)]{henry (\emph{H})}.
Most inductors have inductance best represented in mH or $\mu\text{H}$.
\tags{}

The elemental equation for an inductor is found by taking the time-derivative of the constitutive equation.
\tags{}

\maybeeqn{inductor elemental equation}{eq:elemental_inductor}{
  \begin{align*}
    \frac{d i}{d t} = \frac{1}{L} v
  \end{align*}
}

Inductors store energy in a \emph{magnetic field}.
It is important to notice how inductors are, in a sense, the \emph{opposite} of capacitors.
A capacitor's current is proportional to the time rate of change of its voltage.
An inductor's voltage is proportional to the time rate of change of its current.
\tags{C, V, I}

Inductors are usually made of wire coiled into a number of turns.
The geometry of the coil determines its inductance $L$.
\tags{}

Often, a \keyword{core} material---such as iron and ferrite---is included by wrapping the wire around the core.
This increases the inductance $L$.
\tags{}

Inductors are used extensively in radio-frequency (rf) circuits, with which we won't discuss in this text.
However, they play important roles in ac-dc conversion, filtering, and transformers---all of which we will consider extensively.
\tags{}

The circuit diagram for an inductor is shown in \autoref{fig:inductor}.
\tags{}


\begin{exercises}
\input{ch01_exercises}
\end{exercises}

\end{document}
