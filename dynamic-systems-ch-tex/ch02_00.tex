\documentclass[dynamic_systems.tex]{subfiles}
\begin{document}
\chapter{Linear graph models}
\tags{}
\label{ch:linear_graph_models}

\section{Introduction to linear graphs}
\tags{}

\begin{figure}[!b]
\centering
\includegraphics[width=.85\linewidth]{figures/vignelli.pdf}
\caption{a modern New York subway \href{https://en.wikipedia.org/wiki/File:NYC_subway-4D.svg}{map} in the style of Vignelli (\href{maps.complutense.org}{Jake Berman}).}
\label{fig:vignelli}
\end{figure}

Engineers often use graphical techniques to aid in analysis and design.
We will use \keyword{linear graphs} to represent the \keyword{topology} or structure of a system modeled as interconnected lumped elements.
\tags{}

This represents to us the essential structure of the system in a minimalist form.
In this way, it is like Massimo Vignelli's famous 1972 New York subway system ``map,'' which inspired widespread adoption of his style (see \autoref{fig:vignelli}).\footnote{Vignelli was a brilliant Minimalist designer of many prodcucts, from dishes to clothing, but he was most known for his graphic design. Great places to start studying Vignelli are the documentary \href{http://www.imdb.com/title/tt2610862/}{\emph{Design is One} (2012)} and \href{http://www.vignelli.com/canon.pdf}{\emph{The Vignelli Canon}}.}
Besides minimalism, the key idea in Vignelli subway maps is that the details of the tunnels' paths are irrelevant and, in fact, distracting to the person attempting to get from one station to another.
\tags{}

In a similar way, a linear graph represents the system in a minimalist style, with only two types of objects:
\begin{enumerate}
	\item A set of \keyword{edges}, each of which represents an energy port associated with a system element.
	Each edge is drawn as an oriented line segment ``\tikz{\draw [branch] (0,0) to [bend left] (1,0);}''.
	\item A set of \keyword{nodes}, each of which represents a point of interconnction among system elements.
	Each node is drawn as a dot ``\tikz{\node[graphnode] at (0,0) {};}''.
\end{enumerate}

\begin{wrapfigure}[7]{r}{0.4\textwidth}
  \centering
	\begin{tikzpicture}
		\coordinate (g) at (-1.5,0);
		\draw (g) pic {groundnode};
		\node[above=3pt] at (g) {$\mathcal{V}_\text{ref}$};
		\node[graphnode] (n1) at (0,0) {};
		\node[above=3pt] at (n1) {$\mathcal{V}_1$};
		\node[below=3pt] at (n1) {node};
		\node[graphnode] (n2) at (2,0) {};
		\node[above=3pt] at (n2) {$\mathcal{V}_2$};
		\node[below=3pt] at (n2) {node};
		\draw[branch] (n1) to [bend left] (n2);
		\draw[->,>=stealth',thick,color=violet,line cap=round,bend angle=22] (.5,0) to[bend left] (1.5,0);
		\node[below=-3pt] at (1,0) {$\mathcal{F}$};
		\node[below=0pt] at (1,1) {edge};
	\end{tikzpicture}
  \caption{\label{fig:edge_nodes} an edge with nodes. The across variable is $\mathcal{V} = \mathcal{V}_1 - \mathcal{V}_2$.}%
\end{wrapfigure}

All edges begin and end at nodes.
The nodes represent locations in the system where distinct across-variable values may be measured.
For example, wires that connect elements are actually \emph{nodes} at which voltage may be measured.
Putting an edge together with nodes, we have \autoref{fig:edge_nodes}.
\tags{V, S}

It is important to note that linear graphs can represent \keyword[nonlinearity]{nonlinear} system elements---the name is a reference to the \emph{lines} used.
\tags{}

It is common to choose a node of the graph as the \keyword{reference node}, to which all across-variables are referenced.
Due to its similarity to the electronic \keyword{ground}, we often use these terms interchangeably.
\tags{}

\autoref{fig:linear_graph_ex_simple} shows how a linear graph can be constructed for a simple RC-circuit.
Note that the wires become nodes, the elements become edges, and the reference node represents the circuit ground.
In a similar manner, we will construct linear graphs of circuits, mechanical translational systems, and mechanical rotational systems.
\tags{R, C, D, A}

\begin{figure}[H]
\centering
\begin{tabular}{rl}
\begin{circuitikz}[]
	\draw
		(0,0) to[voltage source, v=$V_s$] (0,2)
		to[R=$R$, i=$ $] (3,2)
		to[C=$C$, i=$ $] (3,0)
		-- (1.5,0) node[ground]{} -- (0,0);
\end{circuitikz}
\hspace{1em}
&
\hspace{1em}
\begin{tikzpicture}
  \draw (0,0) pic {groundnode};
  \node (To) at (-1.5,2) [graphnode] {};
  \node (Ti) at (1.5,2) [graphnode] {};
  %
  \draw [sourcebranch] (To) to [bend right] node [midway,left,outer sep=3pt,anchor=north east] {$V_s$} (0,0);
  \draw [branch] (To) to [bend left] node [midway,above,outer sep=3pt] {$R$} (Ti);
  % \draw [branch] (To) to [bend right] node [midway,below,outer sep=0pt] {$R_2$} (Ti);
  \draw [branch] (Ti) to [bend left] node [midway,right,outer sep=3pt] {$C$} (0,0);
\end{tikzpicture}
\end{tabular}
\caption{an example of a linear graph representation of an RC-circuit.}
\label{fig:linear_graph_ex_simple}
\end{figure}



\section{Sign convention}
\tags{}
\label{lec:sign_convention}

The \keyword{sign} (positive or negative) of a variable is used to represent an orientation of its physical quantity.
For instance, $-3$ m/s could mean $3$ m/s to the \emph{right} or \emph{left}.
No one can say which is better (right is better).
Deciding how the physical quantity corresponds to the sign of the variable is called \keyword{sign assignment}.
When we use a \keyword{sign convention}, we make the assignment in a conventional manner.
For instance, the sign convention for normal stress is that compression is negative and tension is positive.
\tags{v, S}

Why use a sign convention?
If we follow a convention when constructing a problem, we can use the convention's \keyword[sign interpretation]{interpretation} of the result.
For complicated systems, this helps us keep things straight.
Furthermore, if someone else attempts to understand our work, it is much easier to simply say ``using the standard sign convention, \ldots'' than explaining our own snowflake sign assignment.
However, it is nonetheless true that we can assign signs arbitrarily.
\tags{}

Arbitrary? \keyword{Vive la r\'evolution!}
But wait.
If a \emph{source} is present, we must observe some caution.
A source typically \emph{comes with its own convention}.
For instance, if we hook up a power supply to the circuit with the $+$ and $-$ leads a certain way, unless we want to get very confused, we should probably accommodate that sign.
\tags{}

A sign convention for each of the energy domains we've considered follows.
\tags{}

\subsection{Electronic systems}
\tags{}

We use the \keyword{passive sign convention} of electrical engineering, defined below.
% and illustrated in \autoref{fig:sign_convention_electronic_1}
\tags{}

\begin{Definition}{passive sign convention}{passive_sign_convention}
	Power flowing \emph{in} to a component is considered to be \emph{positive} and power flowing \emph{out} of a component is considered \emph{negative}.
\end{Definition}

\newlength{\foolength}
\setlength{\foolength}{.85ex}
\begin{figure}[tb]
	\centering
	\begin{tabular}{cccc}
	\begin{tikzpicture}
		\coordinate (g) at (-1.5,0);
		\draw (g) pic {groundnode};
		\node[above,font=\footnotesize] at (g) {ground};
		\node[graphnode] (n1) at (0,2) {};
		\node[graphnode] (n2) at (0,0) {};
		\draw[branch] (n1) to [bend left] (n2);
		\draw[->,>=stealth',thick,color=mygreen,line cap=round,bend angle=22] (0,1.5) to[bend left] (0,.5);
		\node[left=-3pt] at (0,1) {$i$};
		\draw[thick,color=mygreen,line cap=round]
			($(n1)+(.2,0)$) -- ++ (.6,0)
			coordinate (n12) -- ++(.15,0);
		\draw[thick,color=mygreen,line cap=round]
			($(n2)+(.2,0)$) -- ++ (.6,0)
			coordinate (n22) -- ++(.15,0);
		\draw[->,thick,color=mygreen,line cap=round,>=stealth']
			(n12) -- node[midway,right,color=black] {$v$} 
			node[midway,right,color=black,anchor=north west,font=\footnotesize] {drop} (n22);
	\end{tikzpicture}
	\hspace{\foolength}
	&
	\hspace{\foolength}
	\begin{tikzpicture}
		\node[graphnode] (n1) at (0,0) {};
		\node[graphnode] (n2) at (0,2) {};
		\draw[branch] (n1) to [bend right] (n2);
		\draw[->,>=stealth',thick,color=mygreen,line cap=round,bend angle=22] (0,.5) to[bend right] (0,1.5);
		\node[left=-3pt] at (0,1) {$i$};
		\draw[thick,color=mygreen,line cap=round]
			($(n1)+(.2,0)$) -- ++ (.6,0)
			coordinate (n12) -- ++(.15,0);
		\draw[thick,color=mygreen,line cap=round]
			($(n2)+(.2,0)$) -- ++ (.6,0)
			coordinate (n22) -- ++(.15,0);
		\draw[->,thick,color=mygreen,line cap=round,>=stealth']
			(n12) -- node[midway,right,color=black] {$v$} 
			node[midway,right,color=black,anchor=north west,font=\footnotesize] {drop} (n22);
	\end{tikzpicture}
	\hspace{\foolength}
	&
	\hspace{\foolength}
	\begin{tikzpicture}
		\node[graphnode] (n1) at (0,2) {};
		\node[graphnode] (n2) at (0,0) {};
		\draw[sourcebranch] (n1) to [bend left] (n2);
		\draw[->,>=stealth',thick,color=mygreen,line cap=round,bend angle=22] (-.28,.5) to[bend right] (-.28,1.5);
		\node[left=4pt] at (0,1) {$i$};
		\draw[thick,color=mygreen,line cap=round]
			($(n1)+(.2,0)$) -- ++ (.6,0)
			coordinate (n12) -- ++(.15,0);
		\draw[thick,color=mygreen,line cap=round]
			($(n2)+(.2,0)$) -- ++ (.6,0)
			coordinate (n22) -- ++(.15,0);
		\draw[->,thick,color=mygreen,line cap=round,>=stealth']
			(n12) -- node[midway,right,color=black] {$v$} 
			node[midway,right,color=black,anchor=north west,font=\footnotesize] {drop} (n22);
	\end{tikzpicture}
	\hspace{\foolength}
	&
	\hspace{\foolength}
	\begin{tikzpicture}
		\node[graphnode] (n1) at (0,0) {};
		\node[graphnode] (n2) at (0,2) {};
		\draw[sourcebranch] (n1) to [bend right] (n2);
		\draw[->,>=stealth',thick,color=mygreen,line cap=round,bend angle=22] (-.28,1.5) to[bend left] (-.28,.5);
		\node[left=4pt] at (0,1) {$i$};
		\draw[thick,color=mygreen,line cap=round]
			($(n1)+(.2,0)$) -- ++ (.6,0)
			coordinate (n12) -- ++(.15,0);
		\draw[thick,color=mygreen,line cap=round]
			($(n2)+(.2,0)$) -- ++ (.6,0)
			coordinate (n22) -- ++(.15,0);
		\draw[->,thick,color=mygreen,line cap=round,>=stealth']
			(n12) -- node[midway,right,color=black] {$v$} 
			node[midway,right,color=black,anchor=north west,font=\footnotesize] {drop} (n22);
	\end{tikzpicture}
	\hspace{\foolength}
	\end{tabular}
	\caption{passive sign convention for electronic systems in terms of voltage $v$ and current $i$. Passive elements are on the left, active on the right.}
	\label{fig:sign_convention_electronic_1}
\end{figure}

Because power $\mathcal{P} = v i$, this implies the current and voltage signs are prescribed by the convention.
For \keyword[passive element]{passive elements}, the electrical potential must drop in the direction of positive current flow.
This means the assumed direction of voltage drop across a passive element must be the same as that of the current flow.
For \keyword[active element]{active elements}, which supply power to the circuit, the converse is true:
the voltage drop and current flow must be in opposite directions.
\autoref{fig:sign_convention_electronic_1} illustrates the possible configurations.
\tags{V, I, S}

When drawing a linear graph of a circuit, for each passive element's edge, draw the arrow beside it pointing in the direction of assumed current flow and voltage drop.
\tags{I, V, S}

The purpose of a sign convention is to help us \keyword[interpretation]{interpret} the signs of our results.
For instance, if, at a given instant, a capacitor has voltage $v_C = 3$ V and current $i_C = -2$ A, we compute $\mathcal{P}_C = -6$ W and we know $6$ W of power is flowing \emph{from} the capacitor into the circuit.
\tags{C, A}

For passive elements, there is no preferred direction of ``assumed'' voltage drop and current flow.
If a voltage or current value discovered by performing a circuit analysis is positive, this means the ``assumed'' and ``actual'' directions are the same. For a negative value, the directions are opposite.
\tags{V, I, S}

For active elements, choose the sign in accordance with the physical situation.
For instance, if a positive terminal of a battery is connected to a certain terminal in a circuit, it ill behooves one to simply say ``but Darling, I'm going to call that negative.''
It's positive whether you like it or not, Nancy.
\tags{}

\subsection{Translational mechanical systems}
\tags{}

\begingroup
\setlength\intextsep{0pt}
\begin{wrapfigure}[4]{r}{0.35\textwidth}
  \centering
	\begin{tikzpicture}
		\node[groundmech,minimum height=1cm,minimum width=.5cm] (g) at (0,0) {};
		\draw (g.south west) -- (g.north west);
		\node[rectangle,draw,thick,rounded corners=.5pt,minimum width=1cm,minimum height=1cm] (m) at (-3,0) {$m$};
		\draw[spring] (g.west) -- (m.east);
		\draw[<-,>=stealth',thick,violet] (m.west) -- ++(-.5,0)
		node[above,black] {$F_s$};
	\end{tikzpicture}
  % \caption{\label{fig:inline_mk_1}}%
\end{wrapfigure}

The following steps can be applied to any translational mechanical system.
We introduce the convention with an inline example.
Consider the simple mechanical system shown at right.
\tags{}
\endgroup

\begingroup
\setlength\intextsep{0pt}
\begin{wrapfigure}[4]{r}{0.35\textwidth}
  \centering
	\begin{tikzpicture}
		\node[groundmech,minimum height=1cm,minimum width=.5cm] (g) at (0,0) {};
		\draw (g.south west) -- (g.north west);
		\node[rectangle,draw,thick,rounded corners=.5pt,minimum width=1cm,minimum height=1cm] (m) at (-3,0) {$m$};
		\draw[spring] (g.west) -- (m.east);
		\draw[->,=>stealth',mygreen,thick] (-3,.7) -- ++(.8,0);
		\draw[<-,>=stealth',thick,violet] (m.west) -- ++(-.5,0)
		node[above,black] {$F_s$};
	\end{tikzpicture}
  % \caption{\label{fig:inline_mk_2}}%
\end{wrapfigure}

\noindent{\sffamily\bfseries coordinate arrow}\quad Assign the sign by drawing a coordinate arrow, as shown at right.
The direction of the arrow is arbitrary, however, if possible, assign the positive direction to match the sources.
If the problem allows, it is best practice to have all sources and the coordinate arrow in the same direction.
\tags{}
\endgroup

\newcommand{\blanklineargraph}{%
	\begin{tikzpicture}[]
		\coordinate (g) at (0,0);
		\draw (g) pic[rotate=90] {groundnode};
		\node[graphnode] (n2) at (-2,0) {};
		\draw[sourcebranchnoarrow,bend angle=80] (g) to[bend left] node[midway,below=7pt,left=7pt,anchor=north east] {$F_s$} (n2);
		\draw[branchnoarrow] (n2) to[] node[midway,above] {$m$} (g);
		\draw[branchnoarrow,bend angle=80] (n2) to[bend left] node[midway,above] {$k$} (g);
	\end{tikzpicture}%
}

\begingroup
\setlength\intextsep{0pt}
\begin{wrapfigure}[7]{r}{0.35\textwidth}
  \centering
  \blanklineargraph{}
  % \caption{\label{fig:inline_mk_3}}%
\end{wrapfigure}

\noindent{\sffamily\bfseries draw linear graph without arrows}\quad There are two nodes with distinct velocities: ground and the mass, as shown at right.
The mass node is always drawn to ground.
The spring connects between the mass and ground.
Finally, the force source connects to the mass, where it is applied, and also connects to ground, which is impervious to it.
\tags{M, v, A, S}

\endgroup

\begingroup
\setlength\intextsep{0pt}
\begin{wrapfigure}[5]{r}{0.35\textwidth}
  \centering
	\ifdefined\ispartial
		\blanklineargraph{}
	\else
		\begin{tikzpicture}[]
			\coordinate (g) at (0,0);
			\draw (g) pic[rotate=90] {groundnode};
			\node[graphnode] (n2) at (-2,0) {};
			\draw[sourcebranchnoarrow,bend angle=80] (g) to[bend left] node[midway,below=7pt,left=7pt,anchor=north east] {$F_s$} (n2);
			\draw[branchnoarrow] (n2) to[] node[midway,above] {$m$} (g);
			\draw[branch,bend angle=80] (n2) to[bend left] node[midway,above] {$k$} (g);
		\end{tikzpicture}%
	\fi
  % \caption{\label{fig:inline_mk_4}}%
\end{wrapfigure}

\noindent{\sffamily\bfseries assign spring and damper directions}\quad On each spring and damper element, define the positive velocity drop and edge arrow to be \emph{in the direction of the coordinate arrow}.
\tags{B, k, D, T}
\endgroup

\begingroup
\setlength\intextsep{0pt}
\begin{wrapfigure}[5]{r}{0.35\textwidth}
  \centering
	\ifdefined\ispartial
		\blanklineargraph{}
	\else
		\begin{tikzpicture}[]
			\coordinate (g) at (0,0);
			\draw (g) pic[rotate=90] {groundnode};
			\node[graphnode] (n2) at (-2,0) {};
			\draw[sourcebranchnoarrow,bend angle=80] (g) to[bend left] node[midway,below=7pt,left=7pt,anchor=north east] {$F_s$} (n2);
			\draw[branch] (n2) to[] node[midway,above] {$m$} (g);
			\draw[branch,bend angle=80] (n2) to[bend left] node[midway,above] {$k$} (g);
		\end{tikzpicture}%
	\fi
  % \caption{\label{fig:inline_mk_5}}%
\end{wrapfigure}

\noindent{\sffamily\bfseries assign mass directions}\quad On each mass element, define the positive velocity drop and edge arrow to be \emph{toward ground}.
Sometimes we dash the latter half of the mass edge in to signify that it is ``virtually'' connected to ground.
\tags{V, S}
\endgroup

\begingroup
\setlength\intextsep{0pt}
\begin{wrapfigure}[8]{r}{0.35\textwidth}
  \centering
	\ifdefined\ispartial
		\blanklineargraph{}
	\else
		\begin{tikzpicture}[]
			\coordinate (g) at (0,0);
			\draw (g) pic[rotate=90] {groundnode};
			\node[graphnode] (n2) at (-2,0) {};
			\draw[sourcebranch,bend angle=80] (g) to[bend left] node[midway,below=7pt,left=7pt,anchor=north east] {$F_s$} (n2);
			\draw[branch] (n2) to[] node[midway,above] {$m$} (g);
			\draw[branch,bend angle=80] (n2) to[bend left] node[midway,above] {$k$} (g);
		\end{tikzpicture}%
	\fi
  % \caption{\label{fig:inline_mk_5}}%
\end{wrapfigure}

\noindent{\sffamily\bfseries assign force source directions}\quad On each force source element, define the positive direction as follows.\\
(ideal) If the force source has the \emph{same} definition of positive as your coordinate arrows, draw it \emph{toward the node of application}. \\
(if needed) If the force source has the \emph{opposite} definition of positive as your coordinate arrow, draw it \emph{away from the node of application}.
\tags{}

\endgroup

\begingroup
\setlength\intextsep{0pt}
% \usetikzlibrary{positioning, shapes}
\begin{wrapfigure}[8]{r}{0.35\textwidth}
  \centering
	\ifdefined\ispartial
		\begin{tikzpicture}[]
			\node[cloud, cloud puffs=15.7, cloud ignores aspect, minimum width=4cm, minimum height=3cm, align=center, draw]  at (-1,0) {};
			\coordinate (g) at (0,0);
			\draw (g) pic[rotate=90] {groundnode};
			\node[graphnode] (n2) at (-2,0) {};
			\draw[sourcebranchnoarrow,bend angle=80] (n2) to[bend right] node[midway,below=7pt,left=7pt,anchor=north east] {$V_s$} (g);
			\draw[branchnoarrow] (n2) to[] node[midway,above] {$m$} (g);
			\draw[branchnoarrow,bend angle=80] (n2) to[bend left] node[midway,above] {$k$} (g);
		\end{tikzpicture}%
	\else
		\begin{tikzpicture}[]
			\node[cloud, cloud puffs=15.7, cloud ignores aspect, minimum width=4cm, minimum height=3cm, align=center, draw] (cloud) at (-1,0) {};
			\coordinate (g) at (0,0);
			\draw (g) pic[rotate=90] {groundnode};
			\node[graphnode] (n2) at (-2,0) {};
			\draw[sourcebranch,bend angle=80] (n2) to[bend right] node[midway,below=7pt,left=7pt,anchor=north east] {$V_s$} (g);
			\draw[branch] (n2) to[] node[midway,above] {$m$} (g);
			\draw[branch,bend angle=80] (n2) to[bend left] node[midway,above] {$k$} (g);
		\end{tikzpicture}%
	\fi
  % \caption{\label{fig:inline_mk_5}}%
\end{wrapfigure}

\noindent{\sffamily\bfseries assign velocity source directions}\quad On each velocity source element, define the positive direction as follows.\\
(ideal) If the velocity source has the \emph{same} definition of positive as your coordinate arrows, draw it \emph{away from the node of application}. \\
(if needed) If the velocity source has the \emph{opposite} definition of positive as your coordinate arrow, draw it \emph{toward the node of application}.
\tags{v, S}
\endgroup

\noindent This convention yields the interpretations of \autoref{tab:sign_convention_mech_trans}.

% Please add the following required packages to your document preamble:
% \usepackage{booktabs}
% \usepackage{multirow}
\begingroup
\newcolumntype{Y}{>{\centering\arraybackslash}X}
\renewcommand{\arraystretch}{1.55}
\begin{table}[]
% \captionsetup{justification=centering}
\caption{interpretation of the translational mechanical system sign convention.}
\label{tab:sign_convention_mech_trans}
\begin{adjustbox}{center}
\begin{tabularx}{6in}{@{}lXX|XX@{}}
\toprule
\multirow{2}{*}{} & \multicolumn{2}{c|}{\bfseries force $f$} & \multicolumn{2}{c}{\bfseries velocity $v$} \\ \cmidrule(lr){2-3}\cmidrule(lr){4-5} 
 & positive $+$ & negative $-$ & positive $+$ & negative $-$ \\ \cmidrule(lr){2-3}\cmidrule(lr){4-5} 
$m$ & force \emph{in} direction of the coordinate arrow & force \emph{opposite} the direction of the coordinate arrow & velocity \emph{in} the coordinate arrow direction & velocity \emph{opposite} the coordinate arrow direction \\
$k$ & \emph{compressive} force & \emph{tensile} force & velocity drops \emph{in} the coordinate arrow direction & velocity drops \emph{opposite} the coordinate arrow direction \\
$B$ & \emph{compressive} force & \emph{tensile} force & velocity drops \emph{in} the coordinate arrow direction & velocity drops \emph{opposite} the coordinate arrow direction \\ \bottomrule
\end{tabularx}
\end{adjustbox}
\end{table}
\endgroup

\examplemaybe{%
	translational mechanical sign convention
}{%
  For the system shown, draw a linear graph and assign signs according to the sign convention.
  \begin{center}
  	\newlength{\noff}
  	\setlength{\noff}{5pt}
  	\begin{tikzpicture}
			\node[groundmech,minimum height=1cm,minimum width=.5cm] (g) at (0,0) {};
			\draw (g.south west) -- (g.north west);
			\node[rectangle,draw,thick,rounded corners=.5pt,minimum width=1cm,minimum height=1cm] (m2) at (-3,0) {$m_2$};
			\node[rectangle,draw,thick,rounded corners=.5pt,minimum width=1cm,minimum height=1cm] (m1) at (-6,0) {$m_1$};
			\draw[damper] (g.west) -- node[above=7pt,anchor=south]{$B_2$} (m2.east);
			\draw[spring] ([yshift=-\noff]m1.north east) --node[above=7pt,anchor=south]{$k$} ([yshift=-\noff]m2.north west);
			\draw[damper] ([yshift=\noff]m2.south west) --node[below=7pt,anchor=north]{$B_1$} ([yshift=\noff]m1.south east);
			\draw[<-,>=stealth',thick,violet] (m1.west) -- ++(-.5,0)
			node[above,black] {$F_s$};
		\end{tikzpicture}
  \end{center}
}{%
%TODO 
linear graph

\vspace{7\baselineskip}

}{%
ex:sign_convention_mech_trans%
}

\subsection{Rotational mechanical systems}
\tags{}

% \renewcommand{\AxisRotator}[1][rotate=0]{%
%     \tikz [x=0.2cm,y=0.4cm,thick,-stealth,#1]
%     \draw[line cap=round,inner sep=0pt,outer sep=0pt,#2] (0,0) arc (330:30:1 and 1) node[midway,rectangle,draw,minimum width=.65cm,minimum height=.2cm,color=black,fill=white,xshift=-2pt,thin,anchor=west]{};%
% }

\begingroup
\setlength\intextsep{0pt}
\begin{wrapfigure}[4]{r}{0.35\textwidth}
  \centering
	\begin{tikzpicture}[]
		\node[groundmech,minimum height=1cm,minimum width=.5cm] (g) at (0,0) {};
		\draw (g.south west) -- (g.north west);
		\node[inertia] (J1) at (-3,0) {$J$};
		\draw[shaftcap] ($(J1.west)+(-.3,0)$)
		coordinate (lefty) -- (J1.west);
		\draw[shaftcap] (J1.east) -- ++(.3,0)
			coordinate (righty);
		\draw[spring] (righty) -- (g.west);
		\draw[<-,>=stealth',thick,violet] (lefty) -- ++(-.5,0)
		node[above,black] {$T_s$};
	\end{tikzpicture}
  % \caption{\label{fig:inline_mk_1}}%
\end{wrapfigure}

The following steps can be applied to any rotational mechanical system.
We introduce the convention with an inline example.
Consider the simple system shown at right.
\tags{}
\endgroup

\begingroup
\setlength\intextsep{0pt}
\begin{wrapfigure}[7]{r}{0.35\textwidth}
  \centering
	\begin{tikzpicture}[]
		\node[groundmech,minimum height=1cm,minimum width=.5cm] (g) at (0,0) {};
		\draw (g.south west) -- (g.north west);
		\node[inertia] (J1) at (-3,0) {$J$};
		\draw[shaftcap] ($(J1.west)+(-.3,0)$)
		coordinate (lefty) -- (J1.west);
		\draw[shaftcap] (J1.east) -- ++(.3,0)
			coordinate (righty);
		\draw[spring] (righty) -- (g.west);
		\draw[<-,>=stealth',thick,violet] (lefty) -- ++(-.5,0)
		node[above,black] {$T_s$};
		\draw[->,=>stealth',mygreen,thick] (-2,.55) -- ++(.8,0);
	\end{tikzpicture}
  % \caption{\label{fig:inline_mk_1}}%
\end{wrapfigure}

\noindent{\sffamily\bfseries coordinate arrow}\quad Assign the sign by drawing a coordinate arrow, as shown at right.
The direction of the arrow is arbitrary, however, if possible, assign the positive direction to match the sources.
If the problem allows, it is best practice to have all sources and the coordinate arrow in the same direction.
The right-hand rule is always implied.
\tags{}
\endgroup

\newcommand{\blanklineargraphrot}{%
	\begin{tikzpicture}[]
		\coordinate (g) at (0,0);
		\draw (g) pic[rotate=90] {groundnode};
		\node[graphnode] (n2) at (-2,0) {};
		\draw[sourcebranchnoarrow,bend angle=80] (g) to[bend left] node[midway,below=7pt,left=7pt,anchor=north east] {$T_s$} (n2);
		\draw[branchnoarrow] (n2) to[] node[midway,above] {$J$} (g);
		\draw[branchnoarrow,bend angle=80] (n2) to[bend left] node[midway,above] {$k$} (g);
	\end{tikzpicture}%
}

\begingroup
\setlength\intextsep{0pt}
\begin{wrapfigure}[7]{r}{0.35\textwidth}
  \centering
  \blanklineargraphrot{}
  % \caption{\label{fig:inline_mk_3}}%
\end{wrapfigure}

\noindent{\sffamily\bfseries draw linear graph without arrows}\quad There are two nodes with distinct velocities: ground and the inertia, as shown at right.
The inertia node is always drawn to ground.
The spring connects between the inertia and ground.
Finally, the torque source connects to the mass, where it is applied, and also connects to ground, which is impervious to it.
\tags{v, k}
\endgroup

\begingroup
\setlength\intextsep{0pt}
\begin{wrapfigure}[6]{r}{0.35\textwidth}
  \centering
	\ifdefined\ispartial
		\blanklineargraphrot{}
	\else
		\begin{tikzpicture}[]
			\coordinate (g) at (0,0);
			\draw (g) pic[rotate=90] {groundnode};
			\node[graphnode] (n2) at (-2,0) {};
			\draw[sourcebranchnoarrow,bend angle=80] (g) to[bend left] node[midway,below=7pt,left=7pt,anchor=north east] {$T_s$} (n2);
			\draw[branchnoarrow] (n2) to[] node[midway,above] {$J$} (g);
			\draw[branch,bend angle=80] (n2) to[bend left] node[midway,above] {$k$} (g);
		\end{tikzpicture}%
	\fi
  % \caption{\label{fig:inline_mk_4}}%
\end{wrapfigure}

\noindent{\sffamily\bfseries assign spring and damper directions}\quad On each inline spring and damper element, define the positive velocity drop and edge arrow to be \emph{in the direction of the coordinate arrow}.
Springs and dampers that aren't inline typically connect to ground, toward which edge arrows should point.
\tags{B, k, v, D, T, S}
\endgroup

\begingroup
\setlength\intextsep{0pt}
\begin{wrapfigure}[5]{r}{0.35\textwidth}
  \centering
	\ifdefined\ispartial
		\blanklineargraphrot{}
	\else
		\begin{tikzpicture}[]
			\coordinate (g) at (0,0);
			\draw (g) pic[rotate=90] {groundnode};
			\node[graphnode] (n2) at (-2,0) {};
			\draw[sourcebranchnoarrow,bend angle=80] (g) to[bend left] node[midway,below=7pt,left=7pt,anchor=north east] {$T_s$} (n2);
			\draw[branch] (n2) to[] node[midway,above] {$J$} (g);
			\draw[branch,bend angle=80] (n2) to[bend left] node[midway,above] {$k$} (g);
		\end{tikzpicture}%
	\fi
  % \caption{\label{fig:inline_mk_5}}%
\end{wrapfigure}

\noindent{\sffamily\bfseries assign inertia directions}\quad On each inertia element, define the positive angular velocity drop and edge arrow to be \emph{toward ground}.
Sometimes we dash the latter half of the inertia edge to signify that it is ``virtually'' connected to ground.
\tags{v, S}
\endgroup

\begingroup
\setlength\intextsep{0pt}
\begin{wrapfigure}[8]{r}{0.35\textwidth}
  \centering
	\ifdefined\ispartial
		\blanklineargraphrot{}
	\else
		\begin{tikzpicture}[]
			\coordinate (g) at (0,0);
			\draw (g) pic[rotate=90] {groundnode};
			\node[graphnode] (n2) at (-2,0) {};
			\draw[sourcebranch,bend angle=80] (g) to[bend left] node[midway,below=7pt,left=7pt,anchor=north east] {$T_s$} (n2);
			\draw[branch] (n2) to[] node[midway,above] {$J$} (g);
			\draw[branch,bend angle=80] (n2) to[bend left] node[midway,above] {$k$} (g);
		\end{tikzpicture}%
	\fi
  % \caption{\label{fig:inline_mk_5}}%
\end{wrapfigure}

\noindent{\sffamily\bfseries assign torque source directions}\quad On each torque source element, define the positive direction as follows.\\
(ideal) If the torque source has the \emph{same} definition of positive as your coordinate arrows, draw it \emph{toward the node of application}. \\
(if needed) If the torque source has the \emph{opposite} definition of positive as your coordinate arrow, draw it \emph{away from the node of application}.
\tags{TO, S}

\endgroup

\begingroup
\setlength\intextsep{0pt}
% \usetikzlibrary{positioning, shapes}
\begin{wrapfigure}[8]{r}{0.35\textwidth}
  \centering
	\ifdefined\ispartial
		\begin{tikzpicture}
			% \node[cloud, cloud puffs=15.7, cloud ignores aspect, minimum width=4cm, minimum height=3cm, align=center, draw] (cloud) at (-1,0) {};
			\coordinate (g) at (0,0);
			\draw (g) pic[rotate=90] {groundnode};
			\node[graphnode] (n2) at (-2,0) {};
			\draw[sourcebranchnoarrow,bend angle=80] (n2) to[bend right] node[midway,below=7pt,left=7pt,anchor=north east] {$\Omega_s$} (g);
			\draw[branchnoarrow] (n2) to[] node[midway,above] {$J$} (g);
			\draw[branchnoarrow,bend angle=80] (n2) to[bend left] node[midway,above] {$k$} (g);
		\end{tikzpicture}%
	\else
		\begin{tikzpicture}
			\coordinate (g) at (0,0);
			% \node[cloud, cloud puffs=15.7, cloud ignores aspect, minimum width=4cm, minimum height=3cm, align=center, draw]  at (-1,0) {};
			\draw (g) pic[rotate=90] {groundnode};
			\node[graphnode] (n2) at (-2,0) {};
			\draw[sourcebranch,bend angle=80] (n2) to[bend right] node[midway,below=7pt,left=7pt,anchor=north east] {$\Omega_s$} (g);
			\draw[branch] (n2) to[] node[midway,above] {$J$} (g);
			\draw[branch,bend angle=80] (n2) to[bend left] node[midway,above] {$k$} (g);
		\end{tikzpicture}%
	\fi
  % \caption{\label{fig:inline_mk_5}}%
\end{wrapfigure}

\noindent{\sffamily\bfseries assign angular velocity source directions}\quad On each angular velocity source element, define the positive direction as follows.\\
(ideal) If the source has the \emph{same} definition of positive as your coordinate arrows, draw it \emph{away from the node of application}. \\
(if needed) If the source has the \emph{opposite} definition of positive as your coordinate arrow, draw it \emph{toward the node of application}.
\tags{v, S}

\endgroup

\noindent This convention yields the interpretations of \autoref{tab:sign_convention_mech_rot}.

\newcommand{\foorope}[1]{\draw[double,double distance=3pt,line cap=round] ($#1-(.15,.15)$) -- ++(.3,.3);}
% Please add the following required packages to your document preamble:
% \usepackage{booktabs}
% \usepackage{multirow}
\newcolumntype{Y}{>{\centering\arraybackslash}X}
\renewcommand{\arraystretch}{1.55}
\begin{table}[bt]
% \captionsetup{justification=centering}
\caption{interpretation of the \hspace{6in} rotational mechanical system sign convention.}
\label{tab:sign_convention_mech_rot}
\begin{adjustbox}{center}
\begin{tabularx}{6in}{@{}lXX|XX@{}}
\toprule
\multirow{2}{*}{} & \multicolumn{2}{c|}{\bfseries torque $T$} & \multicolumn{2}{c}{\bfseries angular velocity $\Omega$} \\ \cmidrule(lr){2-3}\cmidrule(lr){4-5} 
 & positive $+$ & negative $-$ & positive $+$ & negative $-$ \\ \cmidrule(lr){2-3}\cmidrule(lr){4-5} 
$J$ & torque \emph{in} direction of the coordinate arrow & torque \emph{opposite} the direction of the coordinate arrow & velocity \emph{in} the coordinate arrow direction & velocity \emph{opposite} the coordinate arrow direction \\
$k$ & 
wring! \adjustbox{valign=t}{\tikz[yscale=1,xscale=1,rotate=90,transform shape]{\node (w1) at (-.42,0) {\AxisRotator[->,rotate=0,mygreen]};\node (w1) at (1.75,0) {\AxisRotator[<-,rotate=0,mygreen]};\foorope{(0,0)}\foorope{(5pt,0)}\foorope{(10pt,0)}\foorope{(15pt,0)}\foorope{(20pt,0)}\foorope{(25pt,0)}\foorope{(30pt,0)}\foorope{(35pt,0)}\foorope{(40pt,0)}\draw[double,double distance=3pt,line cap=round](-5pt,0)--++(.15,.15);\draw[double,double distance=3pt,line cap=round](-5pt,5pt)--++(0,0);\draw[double,double distance=3pt,line cap=round]($(40pt,0)+(0,-5pt)$)--++(.15,.15);\draw[double,double distance=3pt,line cap=round]($(40pt,0)+(5pt,-5pt)$)--++(0,0);}}\quad \adjustbox{valign=t,padding*=0 -1ex 0 0}{\tikz{\draw[branch](0,0) node[graphnode]{}--++(0,2.5) node[graphnode]{};}} & 
wrong! \adjustbox{valign=t}{\tikz[yscale=1,xscale=-1,rotate=90,transform shape]{\node (w1) at (-.42,0) {\AxisRotator[->,rotate=0,mygreen]};\node (w1) at (1.75,0) {\AxisRotator[<-,rotate=0,mygreen]};\foorope{(0,0)}\foorope{(5pt,0)}\foorope{(10pt,0)}\foorope{(15pt,0)}\foorope{(20pt,0)}\foorope{(25pt,0)}\foorope{(30pt,0)}\foorope{(35pt,0)}\foorope{(40pt,0)}\draw[double,double distance=3pt,line cap=round](-5pt,0)--++(.15,.15);\draw[double,double distance=3pt,line cap=round](-5pt,5pt)--++(0,0);\draw[double,double distance=3pt,line cap=round]($(40pt,0)+(0,-5pt)$)--++(.15,.15);\draw[double,double distance=3pt,line cap=round]($(40pt,0)+(5pt,-5pt)$)--++(0,0);}}\quad \adjustbox{valign=t,padding*=0 -1ex 0 0}{\tikz{\draw[branch](0,0) node[graphnode]{}--++(0,2.5) node[graphnode]{};}} & 
velocity drops \emph{in} the coordinate arrow direction & velocity drops \emph{opposite} the coordinate arrow direction \\
$B$ & 
wring! \adjustbox{valign=t}{\tikz[yscale=1,xscale=1,rotate=90,transform shape]{\node (w1) at (-.42,0) {\AxisRotator[->,rotate=0,mygreen]};\node (w1) at (1.75,0) {\AxisRotator[<-,rotate=0,mygreen]};\foorope{(0,0)}\foorope{(5pt,0)}\foorope{(10pt,0)}\foorope{(15pt,0)}\foorope{(20pt,0)}\foorope{(25pt,0)}\foorope{(30pt,0)}\foorope{(35pt,0)}\foorope{(40pt,0)}\draw[double,double distance=3pt,line cap=round](-5pt,0)--++(.15,.15);\draw[double,double distance=3pt,line cap=round](-5pt,5pt)--++(0,0);\draw[double,double distance=3pt,line cap=round]($(40pt,0)+(0,-5pt)$)--++(.15,.15);\draw[double,double distance=3pt,line cap=round]($(40pt,0)+(5pt,-5pt)$)--++(0,0);}}\quad \adjustbox{valign=t,padding*=0 -1ex 0 0}{\tikz{\draw[branch](0,0) node[graphnode]{}--++(0,2.5) node[graphnode]{};}} & 
wrong! \adjustbox{valign=t}{\tikz[yscale=1,xscale=-1,rotate=90,transform shape]{\node (w1) at (-.42,0) {\AxisRotator[->,rotate=0,mygreen]};\node (w1) at (1.75,0) {\AxisRotator[<-,rotate=0,mygreen]};\foorope{(0,0)}\foorope{(5pt,0)}\foorope{(10pt,0)}\foorope{(15pt,0)}\foorope{(20pt,0)}\foorope{(25pt,0)}\foorope{(30pt,0)}\foorope{(35pt,0)}\foorope{(40pt,0)}\draw[double,double distance=3pt,line cap=round](-5pt,0)--++(.15,.15);\draw[double,double distance=3pt,line cap=round](-5pt,5pt)--++(0,0);\draw[double,double distance=3pt,line cap=round]($(40pt,0)+(0,-5pt)$)--++(.15,.15);\draw[double,double distance=3pt,line cap=round]($(40pt,0)+(5pt,-5pt)$)--++(0,0);}}\quad \adjustbox{valign=t,padding*=0 -1ex 0 0}{\tikz{\draw[branch](0,0) node[graphnode]{}--++(0,2.5) node[graphnode]{};}} & 
velocity drops \emph{in} the coordinate arrow direction & velocity drops \emph{opposite} the coordinate arrow direction \\ \bottomrule
\end{tabularx}
\end{adjustbox}
\end{table}

\examplemaybe{%
	rotational mechanical sign convention
}{%
  For the system shown, draw a linear graph and assign signs according to the sign convention.
  \begin{center}
		\begin{tikzpicture}[]
			\node[inertia] (J1) at (-6,0) {$J_1$};
			\draw[shaftcap] ($(J1.west)+(-.3,0)$)
				coordinate (lefty) -- (J1.west);
			\draw[<-,>=stealth',thick,violet] (lefty) -- ++(-.5,0)
				node[above,black] {$T_s$};
			\draw[bearing] (J1.east) -- node[above=.2] {$B_1$} ++(.7,0)
				coordinate (righty);
			\draw[dragcup] (righty) --node[above=.2] {$B_2$} ++(1,0)
				coordinate (J2l);
			\draw[shaftcap] (J2l) -- ++(.3,0)
				node[inertia,anchor=west] (J2) {$J_2$};
			\draw[bearing] (J2.east) -- node[above=.2] {$B_3$} ++(.7,0)
				coordinate (J2r);
			\draw[spring] (J2r) --node[above=.2] {$k$} ++(2,0)
				coordinate (gc);
			\node[groundmech,minimum height=1cm,minimum width=.5cm,anchor=west] (g) at (gc) {};
			\draw (g.south west) -- (g.north west);
		\end{tikzpicture}
  \end{center}
}{%
%TODO 
linear graph

\vspace{8\baselineskip}

uh huh

}{%
ex:sign_convention_mech_rot%
}

\section{Element interconnection laws}
\tags{}

The interconnections among elements constrain across- and through-variable relationships.
The first element interconnection law requires the concept of a \keyword{contour} ``\tikz{\draw [red,densely dashed,line cap=round,thick] plot [smooth cycle] coordinates {(0,0) (.5,0) (.15,.25)};}'': a closed path that does not self-intersect superimposed over the linear graph.
The first interconnection law is called the \keyword{continuity law}.
\tags{}

\begin{Definition}{continuity law}{}
	The sum of the through-variables that flow on \emph{into} a contour on a linear graph is zero, or, in terms of generalized through-variables $\mathcal{F}_i$ for $N$ elements with through variables defined as positive into the contour, 
	\begin{align}
		\sum_{i=1}^N \mathcal{F}_i = 0.
	\end{align}
\end{Definition}

Contours can enclose any number of nodes and edges, as illustrated in \autoref{fig:contours}.
\keyword[KCL]{Kirchhoff's current law} (KCL) is the special case of the continuity law for electronic systems.
\tags{}

\begin{figure}[b]
  \centering
  \begin{tabular}{ccc}
	\begin{tikzpicture}
	  \node (g) at (0,0) [graphnode] {};
	  \node (To) at (-1.5,2) [graphnode] {};
	  \node (Ti) at (1.5,2) [graphnode] {};
	  \draw [sourcebranch] (To) to [bend right] node [midway,left,outer sep=5pt,anchor=north east] {$S$} (g);
	  \draw [branch] (To) to [bend left] node [midway,above,outer sep=3pt] {$1$} (Ti);
	  \draw [branch] (To) to [bend right] node [midway,above,outer sep=3pt] {$2$} (Ti);
	  \draw [branch] (Ti) to [bend left] node [midway,right,outer sep=3pt] {$3$} (g);
	  \node at (0,-.75) {$-\mathcal{F}_1 - \mathcal{F}_2 - \mathcal{F}_S = 0$};
	  \draw[red,densely dashed,line cap=round,thick] (To) circle (10pt);
	\end{tikzpicture}
	&
	\begin{tikzpicture}
	  \node (g) at (0,0) [graphnode] {};
	  \node (To) at (-1.5,2) [graphnode] {};
	  \node (Ti) at (1.5,2) [graphnode] {};
	  \draw [sourcebranch] (To) to [bend right] node [midway,left,outer sep=5pt,anchor=north east] {$S$} (g);
	  \draw [branch] (To) to [bend left] node [midway,above,outer sep=3pt] {$1$} (Ti);
	  \draw [branch] (To) to [bend right] node [midway,above,outer sep=3pt] {$2$} (Ti);
	  \draw [branch] (Ti) to [bend left] node [midway,right,outer sep=3pt] {$3$} (g);
	  \node at (0,-.75) {$-\mathcal{F}_S - \mathcal{F}_3 = 0$};
	  \coordinate (foo) at (0,2);
	  \draw [red,densely dashed,line cap=round,thick] plot [smooth cycle] coordinates {($(foo)+(-2,0)$) ($(foo)+(0,-.8)$) ($(foo)+(2,0)$) ($(foo)+(0,1.1)$)};
	\end{tikzpicture}
	&
	\begin{tikzpicture}
	  \node (g) at (0,0) [graphnode] {};
	  \node (To) at (-1.5,2) [graphnode] {};
	  \node (Ti) at (1.5,2) [graphnode] {};
	  \draw [sourcebranch] (To) to [bend right] node [midway,left,outer sep=5pt,anchor=north east] {$S$} (g);
	  \draw [branch] (To) to [bend left] node [midway,above,outer sep=3pt] {$1$} (Ti);
	  \draw [branch] (To) to [bend right] node [midway,above,outer sep=3pt] {$2$} (Ti);
	  \draw [branch] (Ti) to [bend left] node [midway,right,outer sep=3pt] {$3$} (g);
	  \node at (0,-.75) {$\mathcal{F}_1 + \mathcal{F}_2 + \mathcal{F}_S = 0$};
	  \coordinate (foo) at (Ti);
	  \coordinate (bar) at (g);
	  \draw [red,densely dashed,line cap=round,thick] plot [smooth cycle] coordinates {($(bar)+(-.25,-.25)$) ($(foo)+(.1,.5)$) ($(bar)+(1.75,.65)$)};
	\end{tikzpicture}
	\end{tabular}
  \caption[illustration of different contours.]{illustration of different contours, denoted with red dashed lines ``\tikz{\draw [red,densely dashed,line cap=round,thick] plot [smooth cycle] coordinates {(0,0) (.5,0) (.15,.25)};},'' contours for which the continuity law applies, as shown below each graph.}
  \label{fig:contours}
\end{figure}

The second interconnection law we consider requires the concept of a \keyword{loop} ``\tikz{\draw[branchnoarrow,color=violet,very thick] plot [smooth cycle] coordinates {(0,0) (.5,0) (.35,.25)};}'': a continuous series of edges that begin and end at the same node, not reusing any edges.\footnote{Technically, we need not restrict the definition to series that do not reuse edges for purposes of the compatibility law, but these loops are superfluous and we exclude them here.}
The second interconnection law is called the \keyword{compatibility law}.
\tags{}

\begin{Definition}{compatibility law}{}
	The sum of the across-variable drops on edges around any closed loop on a linear graph is zero, or, in terms of generalized across variables $\mathcal{V}_i$ for $N$ elements in a loop, 
	\begin{align}
		\sum_{i=1}^N \mathcal{V}_i = 0.
	\end{align}
\end{Definition}

\begin{figure}[t]
  \centering
  \begin{tabular}{ccc}
	\begin{tikzpicture}
	  \node (g) at (0,0) [graphnode] {};
	  \node (To) at (-1.5,2) [graphnode] {};
	  \node (Ti) at (1.5,2) [graphnode] {};
	  \draw [sourcebranch] (To) to [bend right] node [midway,left,outer sep=5pt,anchor=north east] {$S$} (g);
	  \draw [branch,color=violet,very thick] (To) to [bend left] node [midway,above,outer sep=3pt] {$1$} (Ti);
	  \draw [branch,color=violet,very thick] (To) to [bend right] node [midway,above,outer sep=3pt] {$2$} (Ti);
	  \draw [branch] (Ti) to [bend left] node [midway,right,outer sep=3pt] {$3$} (g);
	  \node at (0,-.75) {$\mathcal{V}_1 - \mathcal{V}_2 = 0$};
	\end{tikzpicture}
	&
	\begin{tikzpicture}
	  \node (g) at (0,0) [graphnode] {};
	  \node (To) at (-1.5,2) [graphnode] {};
	  \node (Ti) at (1.5,2) [graphnode] {};
	  \draw [sourcebranch,color=violet,very thick] (To) to [bend right] node [midway,left,outer sep=5pt,anchor=north east] {$S$} (g);
	  \draw [branch] (To) to [bend left] node [midway,above,outer sep=3pt] {$1$} (Ti);
	  \draw [branch,color=violet,very thick] (To) to [bend right] node [midway,above,outer sep=3pt] {$2$} (Ti);
	  \draw [branch,color=violet,very thick] (Ti) to [bend left] node [midway,right,outer sep=3pt] {$3$} (g);
	  \node at (0,-.75) {$\mathcal{V}_2 + \mathcal{V}_3 - \mathcal{V}_S = 0$};
	\end{tikzpicture}
	&
	\begin{tikzpicture}
	  \node (g) at (0,0) [graphnode] {};
	  \node (To) at (-1.5,2) [graphnode] {};
	  \node (Ti) at (1.5,2) [graphnode] {};
	  \draw [sourcebranch,color=violet,very thick] (To) to [bend right] node [midway,left,outer sep=5pt,anchor=north east] {$S$} (g);
	  \draw [branch,color=violet,very thick] (To) to [bend left] node [midway,above,outer sep=3pt] {$1$} (Ti);
	  \draw [branch] (To) to [bend right] node [midway,above,outer sep=3pt] {$2$} (Ti);
	  \draw [branch,color=violet,very thick] (Ti) to [bend left] node [midway,right,outer sep=3pt] {$3$} (g);
	  \node at (0,-.75) {$\mathcal{V}_1 + \mathcal{V}_3 - \mathcal{V}_S = 0$};
	\end{tikzpicture}
	\end{tabular}
  \caption[illustration of different loops.]{illustration of different loops, denoted with violet edges ``\tikz{\draw[branch,color=violet,very thick] (0,0) to[bend left] (1,0);},'' loops for which the compatibility law applies.}
  \label{fig:inner_outer_loops}
\end{figure}
A loop can be ``inner'' or ``outer,'' as shown in \autoref{fig:inner_outer_loops}.
\keyword[KVL]{Kirchhoff's voltage law} (KVL) is the special case of the compatibility law for electronic systems.

\examplemaybe{%
	element interconnection laws
}{%
  For the system shown, (a) write three unique continuity and three unique compatibility equations.
  Moreover, (b) write a continuity equation solved for $\mathcal{F}_4$ in terms of $\mathcal{F}_S$ and $\mathcal{F}_1$.
  Finally, (c) write a compatibility equation solved for $\mathcal{V}_5$ in terms of $\mathcal{V}_S$, $\mathcal{V}_3$, and $\mathcal{V}_4$.
  \begin{center}
	\begin{tikzpicture}
	  \node (g) at (0,0) [graphnode] {};
	  \node (To) at (-1.5,2) [graphnode] {};
	  \node (Ti) at (1.5,2) [graphnode] {};
	  \node (n3) at (-3,0) [graphnode] {};
	  \draw [branch] (To) to [bend right] node [midway,left,outer sep=3pt,anchor=north east] {$1$} (g);
	  \draw [branch] (To) to [bend left] node [midway,above,outer sep=3pt] {$2$} (Ti);
	  \draw [branch] (To) to [bend right] node [midway,above,outer sep=3pt] {$3$} (Ti);
	  \draw [branch] (Ti) to [bend left] node [midway,right,outer sep=3pt] {$4$} (g);
	  \draw [branch] (n3) to [bend right] node [midway,below,outer sep=3pt] {$5$} (g);
	  \draw [sourcebranch] (To) to [bend right] node [midway,left,outer sep=5pt,anchor=south east] {$S$} (n3);
	  % \node at (0,-.75) {$\mathcal{F}_1 + \mathcal{F}_2 + \mathcal{F}_S = 0$};
	  % \coordinate (foo) at (Ti);
	  % \coordinate (bar) at (g);
	  % \draw [red,densely dashed,line cap=round,thick] plot [smooth cycle] coordinates {($(bar)+(-.25,-.25)$) ($(foo)+(.1,.5)$) ($(bar)+(1.75,.65)$)};
	\end{tikzpicture}
  \end{center}
}{%
%TODO 
linear graphs
\vspace{25\baselineskip}
and more
}{%
ex:element_interconnection_laws_01%
}

\section{Systematic linear graph modeling}
\tags{}

A \keyword{system graph} is a representation of a physical system as a set of interconnected linear graph elements.
The construction of a system graph requires a number of engineering decisions.
In general, we can use the following procedure.
\tags{}

\begin{enumerate}
	\item Define the system boundary and analyze the physical system to determine the essential features that must be included in the model, especially:
	\begin{enumerate}
		\item inputs,
		\item outputs,
		\item energy domains, and
		\item key elements.
	\end{enumerate}
	\item Form a schematic model of the physical system and assign schematic signs according to the sign convention of \autoref{lec:sign_convention}.
	\item Determine the necessary lumped-parameter elements representing the system's
	\begin{enumerate}
		\item energy sources,
		\item energy storage, and
		\item energy dissipation.
	\end{enumerate}
	\item Identify the across-variables that define the linear graph nodes and draw a set of nodes.
	\item Determine appropriate nodes for each lumped element and include each element in the graph.
	\item Assign linear graph signs according to the sign convention of \autoref{lec:sign_convention}.
\end{enumerate}

The first three of these steps are the hardest.
Considerable physical insight is required to construct an effective model.
Often it is helpful---if not necessary---to have experimental results to guide the process.
\tags{}

\examplemaybe{%
	linear graph model of translational mechanical system
}{%
  For the system shown, develop a linear graph model.
  \begin{center}
	\includegraphics[width=.6\linewidth]{figures/linear_graph_cantilever.pdf}
  \end{center}
}{%
%TODO 
linear graphs

\vspace{7\baselineskip}

}{%
ex:linear_graph_mech_trans_01%
}

\examplemaybe{%
	linear graph model of rotational mechanical system
}{%
  For the system shown, develop a linear graph model.
  \begin{center}
	\includegraphics[width=.8\linewidth]{figures/linear_graph_rotational_example.pdf}
  \end{center}
}{%
%TODO 
linear graphs

\vspace{7\baselineskip}

and more

}{%
ex:linear_graph_mech_rot_01%
}

\examplemaybe{%
	linear graph model of electronic system
}{%
  For the system shown, develop a linear graph model.
  \begin{center}
	\includegraphics[width=1\linewidth]{figures/linear_graph_electronic_ex.pdf}
  \end{center}
}{%
%TODO 
linear graph

\vspace{10\baselineskip}

something

}{%
ex:linear_graph_elec_01%
}

%\section{Physical source modeling} %TODO

\end{document}