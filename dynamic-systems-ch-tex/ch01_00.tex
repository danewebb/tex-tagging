\documentclass[dynamic_systems.tex]{subfiles}
\begin{document}
\chapter{Introduction}
\tags{}
\label{ch:introduction}

Read \cite{Rowell1997}.

\section{The systems approach}
\tags{}

Simon Ramo and Richard Booton, Jr.---the folks who brought us the intercontinental ballistic missile (ICBM) (thanks? \ldots I mean thanks. But, thanks?)---defined \keyword{systems engineering} to be
\tags{}

\begin{quote}
  the design of the whole as distinguished from the design of the parts. \citep{Ramo1984}
\end{quote}
Like the ICBM, many modern technologies require this systems engineering design approach.

A key aspect of the systems engineering design process is the \keyword{mathematical modeling} of the system.
Most systems of interest to the mechanical engineer are \keyword{dynamic systems}: those that change with time.
Enter the time-derivative.
And differential equations.
\tags{}

Dynamic systems exhibit behavior that can be characterized through analysis and called the system's \keyword[system properties]{properties}.
A property of a dynamic system might be how long it takes for it to respond to a given input or which types of inputs would cause a damaging response.
Clearly, such properties are of significant interest to the design process.
\tags{}

\keyword[system dynamics]{System dynamics} is the study and practice of dynamic system analysis and design.
We primarily focus on analysis, at first, since it is a prerequisite for design.
One cannot make a thing do what one wants without first understanding how the thing works.
\tags{}

This Part of the text focuses on \keyword{electromechanical systems}: systems with an interface between electronics and mechanical subsystems.
These are ubiquitous: manufacturing plants, power plants, vehicles, robots, consumer products, anything with a motor---all include electromechanical systems.
In \cref{part:modeling_other_systems}, we will consider more types of systems (e.g.\ fluid and thermal) and their interactions.
\tags{}

Electromechanical systems analysis can proceed with initially separate modeling of the electronic and mechanical subsystems, then, through an unholy union, combining their equations and attempting a solution.
This is fine for simple systems.
However, many systems will require a systematic approach.
\tags{}

We adopt a systematic approach that draws \keyword{graphs} (\'a la \keyword{graph theory}) for electronic and mechanical systems that are intentionally analogous to electronic circuit diagrams.
This allows us to use a single graphical diagram to express a system's composition and interconnections.
Virtually every technique from electronic circuit analysis can be applied to these representations.
Elemental equations, Kirchhoff's laws, impedance---each will be generalized.
In \cref{part:modeling_other_systems}, this same graphical and electronic-analog technique will be extended to other energy domains.
\tags{}

\section{State-determined systems}
\tags{}

A \keyword{system} is defined to be a collection of objects and their relations contained within a \keyword{boundary}.
The collection of those objects that are external to the system and yet interact with it is called the \keyword{environment}.
\keyword[system variables]{System variables} are variables that represent the behavior of the system, both those that are internal to the system and those that are external---that is, with the system's environment.
\tags{}

There are three important classes of system variable, all typically expressed as vector-valued functions of time $t$, conventionally all expressed as column-vectors (and called ``vectors'' even though they're vector-valued functions \ldots because nothing makes sense and we're all going to die).
Consider \autoref{fig:system} for the following definitions.
\keyword[input variables]{Input variables} are system variables that do not depend on the internal dynamics of the system; for a system with $r$ input variables, the ``\keyword{input vector}'' is a vector-valued function $\bm{u}:\mathbb{R}\rightarrow\mathbb{R}^r$.
The environment prescribes inputs, making them \emph{independent variables}.
Conversely, \keyword{output variables} are system variables of interest to the designer; for a system with $m$ output variables, the ``\keyword{output vector}'' is a vector-valued function $\bm{y}:\mathbb{R}\rightarrow\mathbb{R}^m$.
Outputs may or may not directly interact with the environment.
Finally, a minimal set of variables that define the internal state of the system are defined as the \keyword{state variables}; for a system with $n$ state variables, the ``\keyword{state vector}'' is a vector-valued function $\bm{x}:\mathbb{R}\rightarrow\mathbb{R}^n$.
\tags{}

\begin{figure}[b]
	\centering
	\includegraphics[width=1\linewidth]{figures/system.pdf}
	\caption{illustration of a system and its environment.}
	\label{fig:system}
\end{figure}

We consider a very common class of system: those that are \keyword{state-determined}, which are those for which \citep{Rowell1997}
\tags{}

\begin{enumerate}
	\item a mathematical description,
	\item the state at time $t_0$, called the \keyword{initial condition} $\bm{x}(t)|_{t=t_0}$, and
	\item the input $\bm{u}$ for all time $t\ge t_0$
\end{enumerate}

are necessary and sufficient conditions to determine $\bm{x}(t)$ (and therefore $\bm{y}(t)$) for all $t\ge t_0$.
\tags{}

The ``mathematical description'' of the system requires a set of primitive elements be assigned to represent its internal and external interactions. 
The equations derive from two key types of mathematical relationships:
\tags{}
\begin{enumerate}
	\item the input-output behavior of each primitive element and
	\item the topology of interconnections among elements.
\end{enumerate}

The former generate \keyword{elemental equations} and the latter, \keyword[continuity equations]{continuity} or \keyword{compatibility equations}.
\tags{}
\examplemaybe{%
	a state-determined system
}{%
	\begin{minipage}[c]{.52\linewidth}
    In the RC circuit shown, let $V_s$ be a source and $v_o$ the voltage of interest.
    Identify
    \begin{enumerate}
    	\item the system boundary,
    	\item the input vector,
    	\item the output vector,
    	\item a state vector,
    	\item an elemental equation,
    	\item and which equations might be continuity or compatibility equations.
    \end{enumerate}
  \end{minipage}
  \hfill%
  \begin{minipage}[c]{.4\linewidth}
    \begin{circuitikz}[]
			\draw
				(0,0) to[voltage source, v=$V_s$] (0,2)
				to[R=$R$, i=$i_{R}$] (2,2)
				to[C=$C$, i=$i_{C}$] (2,0)
				-- (0,0);
			\draw
				(2,2) to[short, -o] (3,2)
				to[open, v^=$v_o$, -o] (3,0)
				to[short] (2,0);
		\end{circuitikz}
  \end{minipage}
}{%
  \begin{enumerate}
  	\item Everything but $V_s$ and $v_o$.
  	\item $\bm{u}(t) = \begin{bmatrix} V_s(t) \end{bmatrix}$.
  	\item $\bm{y}(t) = \begin{bmatrix} v_o(t) \end{bmatrix}$.
  	\item There is no unique answer, but it turns out the minimimum number of variables that describe the state of an $n$th order system is $n$. $v_C$ is a good choice in this case. So $\bm{x}(t) = \begin{bmatrix} v_C(t) \end{bmatrix}$.
  	\item For instance, $v_R = i_R R$.
  	\item Because they deal with the topology and interconnections of the elements, KCL and KVL are the continuity and compatibility equations.
  \end{enumerate}
}{%
ex:state_determined_systems%
}

\section{Energy, power, and lumping}
\tags{}

The \keyword{law of energy conservation} states that, for an isolated system, the total energy remains constant.
Let $\mathcal{E}:\mathbb{R}\rightarrow\mathbb{R}$ be the function of time representing the total energy in a system and $\mathcal{P}:\mathbb{R}\rightarrow\mathbb{R}$ be the function of time representing \keyword{power} into the system, defined as
\tags{}
\begin{align}
	\mathcal{P}(t) = \frac{d \mathcal{E}(t)}{d t}.
\end{align}

The energy in a system can change if it exchanges energy with its environment.
We consider this exchange to occur through a finite number of \emph{ports}, each of which can supplies or removes energy (positive or negative power), as in \autoref{fig:system_ports}.
This is expressed in an equation for power into a port $\mathcal{P}_i$ and $N$ ports as
\tags{}
\maybeeq{
	\begin{align*}
		\mathcal{P}(t) = \sum_{i=1}^N \mathcal{P}_i(t).
	\end{align*}
}
We construct our systems such that they have no internal energy sources.

\subsection{Lumping}
\tags{}

We have assumed power enters a system via a finite number of ports.
Similarly, we assume the energy \emph{in} a system is stored in a finite number $M$ of distinct elements with energy $\mathcal{E}_i$ such that
\tags{}
\begin{align}
	\mathcal{E}(t) = \sum_{i=1}^M \mathcal{E}_i(t).
\end{align}

We call these elements \keyword{energy storage elements}. Energy can also be dissipated from the system via certain elements called \keyword{energy dissipative elements}.
\tags{B, D}

\begin{figure}[bt]
	\centering
	\includegraphics[width=.7\linewidth]{figures/system_ports.pdf}
	\caption{system ports}
	\label{fig:system_ports}
\end{figure}

Considering a system to have a finite number of elements, as we have done, requires a specific kind of \keyword{abstraction} from real systems.
A familiar example is the ``point mass'' of elementary mechanics.
We say it interacts with its environment via specific connections called ports (maybe it's attached to a spring element) and behaves a certain way in these interactions (for a mass element, Newton's second law).
We do not often encounter an object that behaves as if it has mass, but no volume.
Yet, this is a useful abstraction for many problems.
\tags{}

When we abstract like this, considering an object to be described fully as a discrete object with interaction ports, we are said to be \keyword{lumped-parameter modeling}.
This is often contrasted with \keyword[distributed modeling]{distributed} or \keyword[continuous modeling]{continuous modeling}, which consider the element in greater detail.
For instance, an object might be considered to be distributed through space and perhaps be flexible or behave as a fluid.
\tags{}

It is lumped-\emph{parameter} modeling because we typically define a parameter that governs the behavior of the element, such as resistance or mass. This parameter will enter the system's dynamics via an \keyword{elemental equation} such as Ohm's Law in the case of a resistor or Newton's Second Law in the case of a mass.
\tags{R, D}

Determining if lumped-parameter modeling is proper for a given system is dependent on the type of insight one wants to achieve about the system.
The system itself does not prescribe the proper modeling technique, but the desired understanding does.
Every system is incredibly complex in its behavior, if one considers it at a fine-granularity.
In this light, it is striking that simple models work \emph{at all}.
Nevertheless, lumping is highly effective for many analyses.
\tags{}

It is important to note that lumped-parameter models can be applied at different levels of granularity for the same system.
\keyword[finite element modeling]{Finite element modeling} can use a large number of small lumped-parameter model elements to approximate a continuous model.
Such applications are beyond the scope of this course.
\tags{}

\section{Mechanical translational elements}
\tags{}

We now introduce a few lumped-parameter elements for mechanical systems in translational (i.e.\ straight-line) motion.
Newton's laws of motion can be applied.
Let a \keyword{force} $f$ and \keyword{velocity} $v$ be input to a port in a mechanical translational element.
Since, for mechanical systems, the power into the element is 
\tags{f, v, S}

\begin{align}
	\mathcal{P}(t) &= f(t) v(t)
\end{align}

we call $f$ and $v$ the \keyword{power-flow variables} for mechanical translational systems.
Some mechanical translational elements have two distinct locations between which its velocity is defined (e.g.\ the velocity across a spring's two ends) and other elements have just one (e.g.\ a point-mass), the velocity of which must have an inertial frame of reference.
This is analogous to how a point in a circuit can be said to have a voltage---by which we mean ``relative to ground.''
In fact, we call this mechanical translational inertial reference \keyword{ground}.
\tags{f, v, S}

\keyword[work]{Work} done on the system over the time interval $[0,T]$ is defined as
\begin{align}
	W \equiv \int_0^T \mathcal{P}(\tau) d\tau.
\end{align}
Therefore, the work done on a mechanical system is 
\begin{align}
	W = \int_0^T f(\tau) v(\tau) d\tau.
\end{align}

The \keyword{linear displacement} $x$ is
\begin{align}
	x(t) = \int_0^t v(\tau) d\tau + x(0).
\end{align}
Similarly, the \keyword{linear momentum} is
\begin{align}
	p(t) = \int_0^t f(\tau) d\tau + p(0).
\end{align}

We now consider two elements that can store energy, called \keyword{energy storage elements}; an element that can dissipate energy to a system's environment, called an \keyword[energy dissipative elements]{energy dissipative element}; and two elements that can supply power from outside a system, called \keyword{source elements}.
\tags{B, D}

\subsection{Translational springs}
\tags{}

A \keyword{translational spring} is defined as an element for which the displacement $x$ across it is a monotonic function of the force $f$ through it.
A \keyword{linear translational spring} is a spring for which Hooke's law applies; that is, for which
\tags{f, S}

\begin{align}
\label{eq:hooke}
	f(t) = k x(t),
\end{align}

where $f$ is the force through the spring and $x$ is the displacement across the spring, minus its unstretched length, and $k$ is called the \keyword{spring constant} and is typically a function of the material properties and geometry of the element.
This is called the element's \keyword{constitutive equation} because it constitutes what it means to be a spring.
\tags{f, S, k, T}

% \tikzstyle{spring}= % don't know why I have to repeat this from mybook.sty!
%   [
%     % double,
%     thick,
%     decorate,
%     decoration={coil,pre length=0.5cm,post length=0.5cm,segment length=6,amplitude=6}
%   ]

\begin{wrapfigure}[8]{r}{0.3\textwidth}
  \centering
	\begin{tikzpicture}
		\draw[spring] (0,0) -- node[above=.2] {$k$} (2,0);
		\draw[->] (0,0) -- ++(0,.75)
		-- ++(.3,0)
		node[above] {$v_1$};
		\draw[->] (2,0) -- ++(0,.75)
		-- ++(.3,0)
		node[above] {$v_2$};
		\draw[->] (0,0) -- ++(-.5,0)
		node[left] {$f$};
		\draw[->] (2,0) -- ++(.5,0)
		node[right] {$f$};
		\draw[fill=white] (0,0) circle (2pt);
		\draw[fill=white] (2,0) circle (2pt);
	\end{tikzpicture}
  \caption{\label{fig:spring} schematic symbol for a spring with spring constant $k$ and velocity drop $v = v_1 - v_2$.}%
\end{wrapfigure}

Although there are many examples of nonlinear springs, we can often use a linear model for analysis in some operating regime.
The \keyword{elemental equation} for a linear spring can be found by time-differentiating \autoref{eq:hooke} to obtain
\tags{}

\maybeeq{
\begin{align*}
	\frac{d f}{d t} = k v.
\end{align*}
}

We call this the elemental equation because it relates the element's power-flow variables $f$ and $v$.
\tags{f, v, S}

A spring stores energy as elastic potential energy, making it an \emph{energy storage element}.
The amount of energy it stores depends on the force it applies.
For a linear spring,
\tags{}

\begin{align}
	\mathcal{E}(t) = \frac{1}{2 k} f(t)^2.
\end{align}

\subsection{Point-masses}
\tags{}

A non-relativistic translational point-mass element with mass $m$, velocity $v$ (relative to an inertial reference frame), and momentum $p$ has the constitutive equation
\tags{v, S, M, A}
\begin{align}
	p = m v.
\end{align}
\begin{wrapfigure}[8]{r}{0.35\textwidth}
  \centering
	\begin{tikzpicture}
		\node[rectangle,draw,thick,rounded corners=.5pt,minimum width=1cm,minimum height=1cm] (m) at (0,0) {$m$};
		\node[groundmech,minimum height=1cm] (g) at (-1.5,0) {};
		\draw (g.south east) -- (g.north east);
		\draw[->] (g.north east) -- ++(0,.75)
		-- ++(.3,0)
		node[above] {$v_2$};
		\draw[->] (m.north east) -- ++(0,.75)
		-- ++(.3,0)
		node[above] {$v_1$};
		\draw[->] (m.east) -- ++(.5,0)
		node[right] {$f$};
		% \draw[fill=white] (0,0) circle (2pt);
		% \draw[fill=white] (2,0) circle (2pt);
	\end{tikzpicture}
  \caption{\label{fig:mass} schematic symbol for a point-mass with mass $m$ and velocity drop $v = v_1 - v_2$, where $v_2$ is the constant reference velocity.}%
\end{wrapfigure}
Once again, time-differentiating the constitutive equation gives us the elemental equation:
\maybeeq{
	\begin{align*}
		\frac{d v}{d t} = \frac{1}{m} f,
	\end{align*}
}
\noindent which is just Newton's second law.

Point-masses can store energy (making them \emph{energy storage elements}) in gravitational potential energy or, as will be much more useful in our analyses, in kinetic energy
\tags{M, A}
\begin{align}
	\mathcal{E}(t) = \frac{1}{2} m v^2.
\end{align}

\subsection{Dampers}
\tags{B, D}

% \tikzstyle{damper}= % don't know why I have to repeat this from mybook.sty!
%   [
%     thick,
%     decoration=
%       {
%         markings,
%         mark connection node=dmp,
%         mark=at position 0.5 with
%         {
%           \node (dmp) [thick,inner sep=0pt,transform shape,rotate=-90,minimum width=15pt,minimum height=3pt,draw=none] {};
%           \draw [thick] ($(dmp.north east)+(2pt,0)$) -- (dmp.south east) -- (dmp.south west) -- ($(dmp.north west)+(2pt,0)$);
%           \draw [thick] ($(dmp.north)+(0,-5pt)$) -- ($(dmp.north)+(0,5pt)$);
%         }
%       },
%     decorate
%   ]

\begin{wrapfigure}[10]{r}{0.3\textwidth}
  \centering
	\begin{tikzpicture}
		\draw[damper] (0,0) -- node[above=.3] {$B$} (2,0);
		\draw[->] (0,0) -- ++(0,.75)
		-- ++(.3,0)
		node[above] {$v_1$};
		\draw[->] (2,0) -- ++(0,.75)
		-- ++(.3,0)
		node[above] {$v_2$};
		\draw[->] (0,0) -- ++(-.5,0)
		node[left] {$f$};
		\draw[->] (2,0) -- ++(.5,0)
		node[right] {$f$};
		\draw[fill=white] (0,0) circle (2pt);
		\draw[fill=white] (2,0) circle (2pt);
	\end{tikzpicture}
  \caption{\label{fig:damper} schematic symbol for a damper with damping coefficient $B$ and velocity drop $v = v_1 - v_2$.}%
\end{wrapfigure}

\keyword[dampers]{Dampers} are defined as elements for which the force $f$ through the element is a monotonic function of the velocity $v$ across it.
\keyword[linear dampers]{Linear dampers} have constitutive equation (and, it turns out, elemental equation)
\tags{v, f}

\begin{align}
	f = B v
\end{align}

where $B$ is called the \keyword{damping coefficient}.
Linear damping is often called \keyword{viscous damping} because systems that push viscous fluid through small orifices or those that have lubricated sliding are well-approximated by this equation.
For this reason, a damper is typically schematically depicted as a \keyword{dashpot}.
\tags{}

Linear damping is a reasonable approximation of lubricated sliding, but it is rather poor for \keyword{dry friction} or \keyword{Coulomb friction}, forces for which are not very velocity-dependent.
Aerodynamic \keyword{drag} is quite velocity-dependent, but is rather nonlinear, often represented as
\tags{v, S}
\maybeeq{
	\begin{align*}
		f = c |v| v
	\end{align*}
}
\noindent where $c$ is called the drag constant.

Dampers dissipate energy from the system (typically to heat), making them \emph{energy dissipative elements}.
\tags{}

\subsection{Force and velocity sources}
\tags{S}
%TODO add figures

An \keyword{ideal force source} is an element that provides arbitrary energy to a system via an independent (of the system) force.
The corresponding velocity across the element depends on the system.
\tags{f, v}

An \keyword{ideal velocity source} is an element that provides arbitrary energy to a system via an independent (of the system) velocity.
The corresponding force through the element depends on the system.
\tags{v}


\section{Mechanical rotational elements}
\tags{}

We now introduce a few lumped-parameter elements for mechanical systems in rotational motion.
Newton's laws of motion, in their angular analogs, can be applied.
Let a \keyword{torque} $T$ and \keyword{angular velocity} $\Omega$ be input to a port in a mechanical rotational element.
Since, for mechanical rotational systems, the power into the element is 
\tags{TO, OM, S}

\begin{align}
	\mathcal{P}(t) &= T(t) \Omega(t)
\end{align}

we call $T$ and $\Omega$ the \keyword{power-flow variables} for mechanical rotational systems.
Some mechanical rotational elements have two distinct locations between which its angular velocity is defined (e.g.\ the angular velocity across a spring's two ends) and other elements have just one (e.g.\ a rotational inertia), the velocity of which must have an inertial frame of reference.
This is analogous to how a point in a circuit can be said to have a voltage---by which we mean ``relative to ground.''
In fact, we call this mechanical rotational inertial reference \keyword{ground}.
\tags{V, v, S}

\keyword[work]{Work} done on the system over the time interval $[0,t_f]$ is defined as
\begin{align}
	W \equiv \int_0^{t_f} \mathcal{P}(\tau) d\tau.
\end{align}
Therefore, the work done on a mechanical system is 
\begin{align}
	W = \int_0^{t_f} T(\tau) \Omega(\tau) d\tau.
\end{align}

The \keyword{angular displacement} $\theta$ is
\begin{align}
	\theta(t) = \int_0^t \Omega(\tau) d\tau + \theta(0).
\end{align}
Similarly, the \keyword{angular momentum} is
\begin{align}
	h(t) = \int_0^t T(\tau) d\tau + h(0).
\end{align}

We now consider two elements that can store energy, called \keyword{energy storage elements}; an element that can dissipate energy to a system's environment, called an \keyword[energy dissipative elements]{energy dissipative element}; and two elements that can supply power from outside a system, called \keyword{source elements}.
\tags{J, k, T, A}

\subsection{Rotational springs}
\tags{k, T}

A \keyword{rotational spring} is defined as an element for which the angular displacement $\theta$ across it is a monotonic function of the torque $T$ through it.
A \keyword{linear rotational spring} is a rotational spring for which the angular form of Hooke's law applies; that is, for which
\tags{}

\begin{align}
	\label{eq:hooke2}
	T(t) = k \theta(t),
\end{align}

where $T$ is the torque through the spring and $\theta$ is the angular displacement across the spring and $k$ is called the \keyword{torsional spring constant} and is typically a function of the material properties and geometry of the element.
This is called the element's \keyword{constitutive equation} because it constitutes what it means to be a rotational spring.
\tags{TO}

\begin{wrapfigure}[5]{r}{0.4\textwidth}
  \centering
	\begin{tikzpicture}[every node/.style={inner sep=0,outer sep=0}]
		\draw[spring] (0,0) -- node[above=.3] {$k$} (2,0);
		\node (w1) at (0,0) {\AxisRotator[->,rotate=0,mygreen]};
		\node[above] at (w1.north) {$\Omega_1$};
		\draw[->,thick,color=violet] ($(w1.west)-(.5,0)$) coordinate (lefty) -- (w1.west);
		\node[left=.07] at (lefty) {$T$};
		\node (w2) at (2,0) {\AxisRotator[->,rotate=0,mygreen]};
		\node[above] at (w2.north) {$\Omega_2$};
		\draw[->,thick,color=violet] ($(w2.east)+(.5,0)$) coordinate (righty) -- (w2.east);
		\node[right=.07] at (righty) {$T$};
	\end{tikzpicture}
  \caption{\label{fig:spring} schematic symbol for a spring with spring constant $k$ and angular velocity drop $\Omega = \Omega_1 - \Omega_2$.}%
\end{wrapfigure}

Although there are many examples of nonlinear springs, we can often use a linear model for analysis in some operating regime.
The \keyword{elemental equation} for a linear spring can be found by time-differentiating \autoref{eq:hooke2} to obtain
\tags{}

\maybeeq{
\begin{align*}
	\frac{d T}{d t} = k \Omega.
\end{align*}
}

We call this the elemental equation because it relates the element's power-flow variables $T$ and $\Omega$.
\tags{TO, OM, S}

A rotational spring stores energy as elastic potential energy, making it an \emph{energy storage element}.
The amount of energy it stores depends on the torque it applies.
For a linear rotational spring,
\tags{k, TO, S, T}
\begin{align}
	\mathcal{E}(t) = \frac{1}{2 k} T(t)^2.
\end{align}

\subsection{Moments of inertia}
\tags{}

A \keyword{moment of inertia} element with moment of inertia $J$, angular velocity $\Omega$ (relative to an inertial reference frame), and angular momentum $h$ has the constitutive equation
\tags{J, OM, A, S}
\begin{align}
	h = J\Omega.
\end{align}
\begin{wrapfigure}[5]{r}{0.35\textwidth}
  \centering
	\begin{tikzpicture}[every node/.style={inner sep=0,outer sep=0}]
		\node[rectangle,draw,minimum width=1cm,minimum height=1.5cm] (J1) at (1,0) {$J$};
		\node (w1) at (0,0) {\AxisRotator[->,rotate=0,mygreen]};
		\node[above] at (w1.north) {$\Omega_2$};
		\node[left=.07] at (w1.west) {ref};
		\node[anchor=west] (w2) at (J1.east) {\AxisRotator[->,rotate=0,mygreen]};
		\node[above] at (w2.north) {$\Omega_1$};
		\draw[->,thick,color=violet] ($(w2.east)+(.5,0)$) coordinate (righty) -- (w2.east);
		\node[right=.07] at (righty) {$T$};
	\end{tikzpicture}
  \caption{\label{fig:mass} schematic symbol for a moment of inertia with inertia $J$ and velocity drop $\Omega = \Omega_1 - \Omega_2$, where $\Omega_2$ is a constant reference velocity.}%
\end{wrapfigure}
Once again, time-differentiating the constitutive equation gives us the elemental equation:
\maybeeq{
	\begin{align*}
		\frac{d \Omega}{d t} = \frac{1}{J} T,
	\end{align*}
}
\noindent which is just the angular version of Newton's second law.

Any rotating element with mass can be considered as a lumped-inertia element.
The \keyword{flywheel} is the quintessential example.
Flywheels store energy in their angular momentum, with the expression
\tags{J, A}
\begin{align}
	\mathcal{E}(t) = \frac{1}{2} J \Omega^2,
\end{align}
making them (and all moments of inertia) \emph{energy storage elements}.

\subsection{Rotational dampers}
\tags{B, D}

\begin{wrapfigure}[13]{r}{0.4\textwidth}
  \centering
	\begin{tikzpicture}[every node/.style={inner sep=0,outer sep=0}]
		\draw[dragcup] (0,0) -- node[above=.4] {$B$} (2,0);
		\node (w1) at (0,0) {\AxisRotator[->,rotate=0,mygreen]};
		\node[above] at (w1.north) {$\Omega_1$};
		\draw[->,thick,color=violet] ($(w1.west)-(.5,0)$) coordinate (lefty) -- (w1.west);
		\node[left=.07] at (lefty) {$T$};
		\node (w2) at (2,0) {\AxisRotator[->,rotate=0,mygreen]};
		\node[above] at (w2.north) {$\Omega_2$};
		\draw[->,thick,color=violet] ($(w2.east)+(.5,0)$) coordinate (righty) -- (w2.east);
		\node[right=.07] at (righty) {$T$};
	\end{tikzpicture} \\ \vspace{1\baselineskip}
	\begin{tikzpicture}[every node/.style={inner sep=0,outer sep=0}]
		\draw[bearing] (0,0) -- node[above=.4] {$B$} (2,0);
		\node (w1) at (0,0) {\AxisRotator[->,rotate=0,mygreen]};
		\node[above] at (w1.north) {$\Omega$};
		\draw[->,thick,color=violet] ($(w1.west)-(.5,0)$) coordinate (lefty) -- (w1.west);
		\node[left=.07] at (lefty) {$T$};
		\node (w2) at (2,0) {\AxisRotator[->,rotate=0,white]};
		% \node[above] at (w2.north) {$\Omega_2$};
		\draw[->,thick,color=violet] ($(w2.east)+(.5,0)$) coordinate (righty) -- (w2.east);
		\node[right=.07] at (righty) {$T$};
	\end{tikzpicture}
  \caption{\label{fig:dragcup} schematic symbol for a drag cup (above) and bearing (below) with damping coefficient $B$. For the drag cup, the angular velocity drop is $\Omega = \Omega_1 - \Omega_2$ and for the bearing, $\Omega$ is reference is ground.}%
\end{wrapfigure}

\keyword[rotational dampers]{Rotational dampers} are defined as elements for which the torque $T$ through the element is a monotonic function of the angular velocity $\Omega$ across it.
\keyword[linear rotational dampers]{Linear rotational dampers} have constitutive equation (and, it turns out, elemental equation)
\begin{align}
	T = B \Omega
\end{align}
where $B$ is called the \keyword{torsional damping coefficient}.
Linear torsional damping is often called \keyword{torsional viscous damping} because systems that push viscous fluid through small orifices or those that have lubricated bearings are well-approximated by this equation.
For this reason, a damper is typically schematically depicted as a \keyword{drag cup} or as a \keyword{bearing}, both of which are shown in \autoref{fig:dragcup}.
\tags{}

Linear damping is a reasonable approximation of lubricated sliding, but it is rather poor for \keyword{dry friction} or \keyword{Coulomb friction}, forces for which are not very velocity-dependent.
\tags{OM, S}

Rotational dampers dissipate energy from the system (typically to heat), making them \emph{energy dissipative elements}.
\tags{}

\subsection{Torque and angular velocity sources}
\tags{S}
%TODO add figures

An \keyword{ideal torque source} is an element that provides arbitrary energy to a system via an independent (of the system) torque.
The corresponding angular velocity across the element depends on the system.
\tags{TO, OM}

An \keyword{ideal angular velocity source} is an element that provides arbitrary energy to a system via an independent (of the system) angular velocity.
The corresponding torque through the element depends on the system.
\tags{OM, TO}

\section{Electronic elements}
\tags{}

We now review a few lumped-parameter elements for electronic systems.
Let a \keyword{current} $i$ and \keyword{voltage} $v$ be input to a port in an electronic element.
Since, for electronic system, the power into the element is
\begin{align}
	P(t) = i(t) v(t)
\end{align}
we call $i$ and $v$ the \keyword{power-flow variables}.
Voltage is always understood to be between two points in a circuit.
If only one point is included, the voltage is implicitly relative to a reference voltage, called \keyword{ground}.
\tags{V, I}

The \keyword{magnetic flux linkage} $\lambda$ is 
\begin{align}
	\lambda(t) = \int_0^t v(\tau) d\tau + \lambda{(0)}.
\end{align}
Similarly, the \keyword{charge} is
\begin{align}
	q(t) = \int_0^t i(\tau) d\tau + q(0).
\end{align}

We now consider two elements that can store energy, called \keyword{energy storage elements}; an element that can dissipate energy to a system's environment, called an \keyword{energy dissipative element}; and two elements that can supply power from outside a system, called \keyword{source elements}.
\tags{}

\subsection{Capacitors}
\tags{C, A}

Capacitors have two terminal and are composed of two conductive surfaces separated by some distance.
One surface has charge $q$ and the other $-q$.
A capacitor stores energy in an \emph{electric field} between the surfaces.
\tags{}

Let a capacitor with voltage $v$ across it and charge $q$ be characterized by the parameter \keyword{capacitance} $C$, where the constitutive equation is
\tags{V, S}
\begin{align}
  q = C v.
\end{align}

The capacitance has derived SI unit \keyword[farad (F)]{farad (\emph{F})}, where $\text{F} = \text{A}\cdot\text{s}/\text{V}$.
A farad is actually quite a lot of capacitance.
Most capacitors have capacitances best represented in $\mu\text{F}$, nF, and pF.
\tags{C, A}

The time-derivative of this equation yields the $v$-$i$ relationship (what we call the ``elemental equation'') for capacitors.
\tags{V, I, S}
\begin{align}
  \frac{d v}{d t} = \frac{1}{C}\, i
\end{align}
% Resistors have only algebraic $i$-$v$ relationships, so circuits with only sources and resistors can be described by \emph{algebraic} relationships.
% The dynamics of circuits with capacitors are described with \emph{differential equations}.

\begin{figure}[b]%
  \centering
  \subbottom[\label{fig:bipolar_capacitor} bipolar capacitor.]{%
    \begin{circuitikz}[]
      \draw
        (0,0) to[C=$C$] (2,0);
    \end{circuitikz}
  }%
  \qquad
  \subbottom[\label{fig:polarized_capacitor} polarized capacitor]{%
    \begin{circuitikz}[]
      \draw
        (2,0) to[pC, l_=$C$, v>=$ $] (0,0);
    \end{circuitikz}
  }
  \caption{capacitor circuit diagram symbols.}%
  \label{fig:capacitors}%
\end{figure}

Capacitors allow us to build many new types of circuits: filtering, energy storage, resonant, blocking (blocks dc-component), and bypassing (draws ac-component to ground).
\tags{A}

Capacitors come in a number of varieties, with those with the largest capacity (and least expensive) being \keyword[electrolytic capacitor]{electrolytic} and most common being \keyword[ceramic capacitor]{ceramic}. There are two functional varieties of capacitors: \keyword[bipolar capacitor]{bipolar} and \keyword[polarized capacitor]{polarized}, with circuit diagram symbols shown in \autoref{fig:capacitors}.
Polarized capacitors can have voltage drop across in only one direction, from \keyword{anode} ($+$) to \keyword{cathode} ($-$)---otherwise they are damaged or may \keyword[explosion]{explode}.
Electrolytic capacitors are polarized and ceramic capacitors are bipolar.
\tags{A}

So what if you need a high-capacitance bipolar capacitor?
Here's a trick: place identical high-capacity polarized capacitors \keyword{cathode-to-cathode}.
What results is effectively a bipolar capacitor with capacitance \emph{half} that of one of the polarized capacitors.
\tags{}

\subsection{Inductors}
\tags{L, T}

\begin{wrapfigure}{R}{0.3\textwidth}
  \centering
  \begin{circuitikz}[]
    \draw
      (0,0) to[L=$L$] (2,0);
  \end{circuitikz}
  \caption{\label{fig:inductor} inductor circuit diagram symbol.}%
\end{wrapfigure}

A \keyword{pure inductor} is defined as an element in which \keyword{flux linkage $\lambda$}---the integral of the voltage---across the inductor is a monotonic function of the current $i$.
An \keyword{ideal inductor} is such that this monotonic function is linear, with slope called the \keyword{inductance $L$}; i.e.\ the ideal constitutive equation is
\tags{I, V, S}
\begin{align}
  \lambda = L i
\end{align}

The units of inductance are the SI derived unit \keyword[henry (H)]{henry (\emph{H})}.
Most inductors have inductance best represented in mH or $\mu\text{H}$.
\tags{}

The elemental equation for an inductor is found by taking the time-derivative of the constitutive equation.
\tags{}
\begin{align}
  \frac{d i}{d t} = \frac{1}{L} v
\end{align}

Inductors store energy in a \emph{magnetic field}.
It is important to notice how inductors are, in a sense, the \emph{opposite} of capacitors.
A capacitor's current is proportional to the time rate of change of its voltage.
An inductor's voltage is proportional to the time rate of change of its current.
\tags{C, I, V, A, S}

Inductors are usually made of wire coiled into a number of turns.
The geometry of the coil determines its inductance $L$.
\tags{}

Often, a \keyword{core} material---such as iron and ferrite---is included by wrapping the wire around the core.
This increases the inductance $L$.
\tags{}

Inductors are used extensively in radio-frequency (rf) circuits, which we won't discuss in this text.
However, they play important roles in ac-dc conversion, filtering, and transformers---all of which we will consider extensively.
\tags{}

The circuit diagram for an inductor is shown in \autoref{fig:inductor}.
\tags{}

\subsection{Resistors}
\tags{R, D}

\begin{wrapfigure}{R}{0.3\textwidth}
  \centering
  \begin{circuitikz}[]
    \draw
      (0,0) to[R=$R$] (2,0);
  \end{circuitikz}
  \caption{\label{fig:resistor} resistor circuit diagram symbol.}%
\end{wrapfigure}

Resistors \emph{dissipate} energy from the system, converting electrical energy to thermal energy (heat).
The \keyword{constitutive equation} for an ideal resistor is
\begin{align}
	v = i R.
\end{align}

This is already in terms of power variables, so it is also the \keyword{elemental equation}.
\tags{}

\subsection{Sources}
\tags{S}
Sources (a.k.a.\ supplies) supply power to a circuit. There are two primary types: \emph{voltage sources} and \emph{current sources}.
\tags{V, I}

\subsubsection{Ideal voltage sources}
\tags{V}

An ideal voltage source provides exactly the voltage a user specifies, independent of the circuit to which it is connected.
All it must do in order to achieve this is to supply whatever current necessary.
\tags{V, I}

\subsubsection{Ideal current sources}
\tags{I}

An ideal current source provides exactly the current a user specifies, independent of the circuit to which it is connected.
All it must do in order to achieve this is to supply whatever voltage necessary.
\tags{}

\subsubsection{Modeling real sources}
\tags{}

No real source can produce infinite power.
Some have feedback that controls the output within some finite power range.
These types of sources can be approximated as ideal when operating within its specifications.
Many voltage sources (e.g.\ batteries) do not have internal feedback controlling the voltage.
When these sources are ``loaded'' (delivering power) they cannot maintain their nominal output, be that voltage or current.
We model these types of sources as ideal sources in series or parallel with a resistor, as illustrated in \autoref{fig:real_sources}.
\tags{V, I, R, D}

\begin{figure}[b]%
  \centering
  \subbottom[\label{fig:voltage_source_real} real voltage source model.]{%
    \includegraphics{figures/voltage_source_real.pdf}
  }%
  \qquad
  \subbottom[\label{fig:voltage_source_real} real current source model.]{%
    \includegraphics{figures/current_source_real.pdf}
  }
  \caption{Models for power-limited ``real'' sources.}%
  \label{fig:real_sources}%
\end{figure}

Most manufacturers specify the nominal resistance of a source as the ``output resistance.''
A typical value is $50\ \Omega$.
\tags{R, D}

\section{Generalized through- and across-variables}
\tags{}
%TODO combine with next lecture

We have considered mechanical translational, mechanical rotational, and electronic systems---which we refer to as different \keyword{energy domains}.
There are analogies among these systems that allow for generalizations of certain aspects of these systems.
These generalizations will allow us to use a single framework for unifying the analysis of these (and other) dynamic systems.
\tags{}

There are two important classes of variables common to lumped-parameter dynamic systems: \emph{across-variables} and \emph{through-variables}.
\tags{}

An \keyword{across-variable} is one that makes reference to two nodes of a system element.
For instance, the following are across-variables:
\tags{}

\begin{itemize}
	\item \mayb{voltage $v$,}
	\item \mayb{velocity $v$, and}
	\item \mayb{angular velocity $\Omega$.}
\end{itemize}
We denote a \keyword{generalized across-variable} as $\mathcal{V}$.

A \keyword{through-variable} is one that represents a quantity that passes through a system element.
For instance, the following are through-variables:
\tags{}
\begin{itemize}
	\item \mayb{current $i$,}
	\item \mayb{force $f$, and}
	\item \mayb{torque $T$.}
\end{itemize}
We denote a \keyword{generalized through-variable} as $\mathcal{F}$.

The \keyword{generalized integrated across-variable} $\mathcal{X}$ is
\begin{align}
	\mathcal{X} = \int_0^t \mathcal{V}(\tau) d\tau + \mathcal{X}(0).
\end{align}

The \keyword{generalized integrated through-variable} $\mathcal{H}$ is
\begin{align}
	\mathcal{H} = \int_0^t \mathcal{F}(\tau) d\tau + \mathcal{H}(0).
\end{align}

For mechanical and electronic systems, power $\mathcal{P}$ passing through a lumped-parameter element is 
\begin{align}
	\mathcal{P}(t) = \mathcal{F}(t) \mathcal{V}(t).
\end{align}

These generalized across- and through-variables are sometimes used in analysis.
However, the key idea here is that there are two classes of power-flow variables: across and through.
These two classes allow us to strengthen the sense in which we consider different dynamic systems to be analogous.
\tags{}

\section{Generalized one-port elements}
\tags{}

We can categorize the behavior of one-port elements---electronic, mechanical translational, and mechanical rotational---considered thus far.
In the following sections, we consider two types of energy storage elements, dissipative elements, and source elements.
\tags{}

\subsection{A-type energy storage elements}
\tags{A}

An element that stores energy as a function of its across-variable is called an \keyword{A-type energy storage element}.
Sometimes we call it a \keyword{generalized capacitor} because a capacitor is an A-type energy storage element.
\tags{C}

For generalized through-variable $\mathcal{F}$, across-variable $\mathcal{V}$, integrated through-variable $\mathcal{H}$, and integrated across-variable $X$ the ideal, linear constitutive equation is
\begin{align}\label{eq:generalized_con_A}
	\mathcal{H} = C \mathcal{V}
\end{align}
for $C\in\mathbb{R}$ called the \keyword[generalized capacitance $C$]{generalized capacitance}.
Differentiating \autoref{eq:generalized_con_A} with respect to time, the elemental equation is
\maybeeq{
\begin{align*}
	\frac{d \mathcal{V}}{d t} = \frac{1}{C} \mathcal{F}.
\end{align*}
}

A-type energy storage elements considered thus far are \keyword{capacitors}, \keyword[masses]{translational masses}, and \keyword[rotational inertia]{rotational moments of inertia}.
As with generalized variables, the analogs among elements are more important than are generalized A-type energy storage elements.
\tags{C}

\subsection{T-type energy storage elements}
\tags{T}

An element that stores energy as a function of its through-variable is called a \keyword{T-type energy storage element}.
Sometimes we call it a \keyword{generalized inductor} because an inductor is a T-type energy storage element.
\tags{L}

The ideal, linear constitutive equation is
\begin{align}\label{eq:generalized_con_T}
	\mathcal{X} = L \mathcal{F}
\end{align}
for $L\in\mathbb{R}$ called the \keyword[generalized inductance $L$]{generalized inductance}.
Differentiating \autoref{eq:generalized_con_T} with respect to time, the elemental equation is
\tags{L}
\maybeeq{
\begin{align*}
	\frac{d \mathcal{F}}{d t} = \frac{1}{L} \mathcal{V}.
\end{align*}
}

T-type energy storage elements considered thus far are \keyword{inductors}, \keyword[translational springs]{translational springs}, and \keyword[rotational springs]{rotational springs}.
As with generalized variables, the analogs among elements are more important than are generalized T-type energy storage elements.
\tags{L, k}

\subsection{D-type energy dissipative elements}
\tags{D}

An element that dissipates energy from the system and has an algebraic relationship between its through-variable and its across-variable is called a \keyword{D-type energy dissipative element}.
Sometimes we call it a \keyword{generalized resistor} because a resistor is a D-type energy dissipative element.
\tags{R, B}

The ideal, linear constitutive and elemental equation is
\begin{align}\label{eq:generalized_con_D}
	\mathcal{V} = R \mathcal{F}
\end{align}
for $R\in\mathbb{R}$ called the \keyword[generalized resistance $R$]{generalized resistance}.

D-type energy dissipative elements considered thus far are \keyword{resistors}, \keyword[translational dampers]{translational dampers}, and \keyword[rotational dampers]{rotational dampers}.
As with generalized variables, the analogs among elements are more important than are generalized D-type energy dissipative elements.
\tags{R, B}

\subsection{Sources}
\tags{S}

An \keyword{ideal through-variable source} is an element that provides arbitrary energy to a system via an independent (of the system) through-variable.
The corresponding across-variable depends on the system.
Current, force, and torque sources are the through-variable sources considered thus far.
\tags{I, f, TO}

An \keyword{ideal across-variable source} is an element that provides arbitrary energy to a system via an independent (of the system) across-variable.
The corresponding through-variable depends on the system.
Voltage, translational velocity, and angular velocity are the across-variable sources considered thus far.
\tags{V, v, OM}

\end{document}