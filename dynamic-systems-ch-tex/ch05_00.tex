\documentclass[dynamic_systems.tex]{subfiles}
\begin{document}
%TODO
\chapter[Superposition, stability, etc.]{Superposition, stability, and other LTI system properties}
\tags{}

In this chapter, we will extend our understanding of linear, time-invariant (LTI) system properties.
We must keep in mind a few important definitions.
\tags{}

The \keyword{transient response} of a system is its response during the initial time-interval during which the initial conditions dominate.
The \keyword{steady-state response} of a system is its remaining response, which occurs after the transient response.
\cref{fig:transient_steady_responses} illustrates these definitions.
\tags{}

The \keyword{free response} of a system is the response of the system to initial conditions---\emph{without forcing} (i.e.\ the specific solution of the io ODE with the forcing function identically zero).
This is closely related to, but distinct from, the transient response, which is the free response \emph{plus} an additional term.
This additional term is the \keyword{forced response}: the response of the system to a forcing function---\emph{without initial conditions} (i.e.\ the specific solution of the io ODE with the initial conditions identically zero).
\tags{}

\begin{figure}[H]
\centering
\input{figures/transient_steady_responses_edit.tex}
\vspace{-.5\baselineskip}
\caption{transient and steady-state responses. Note that the transition is not precisely defined. (Figure adapted from \emph{Electronics: an introduction} by Picone.)}
\label{fig:transient_steady_responses}
\end{figure}

\section[Superposition, derivative, and integral properties]{Superposition, derivative, and integral properties of LTI systems}
\tags{}
\label{lec:LTI_properties}

\input{../_common/lti_properties}

\section[Equilibrium and stability properties]{Equilibrium and stability properties of LTI systems}
\tags{}

For a system with LTI state-space model $\dot{\bm{x}} = A \bm{x} + B \bm{u}$, $\bm{y} = C\bm{x}+D\bm{u}$, the model is in \keyword{equilibrium} state $\bm{x}_0$ if all time-derivatives of the state vector ($\dot{\bm{x}}$, $\ddot{\bm{x}}$, $\dddot{\bm{x}}$, $\cdots$) are zero $\bm{0}$.
This implies $A \bm{x}_0 + B \bm{u} = 0$.
For constant input $\overline{\bm{u}}$, this implies
\tags{}
\maybeeq{
	\begin{align*}
		A\bm{x}_0 = -B\overline{\bm{u}}.
	\end{align*}
}

\noindent If $A$ is invertible,\footnote{If $A$ is not invertible, the system has at least one eigenvalue equal to zero, which yields an \emph{equilibrium subspace} equal to an offset (by $B\overline{\bm{u}}$) of the null space of the state space $\mathbb{R}^n$.} as is often the case, there is a unique solution for a single \keyword{equilibrium} state:
\tags{}
\maybeeq{
\begin{align*}
	\bm{x}_0 &= -A^{-1} B \overline{\bm{u}}.
\end{align*}
}

If $\bm{x}$ is perturbed from an equilibrium state $\bm{x}_0$, the response $\bm{x}(t)$ can:
\begin{enumerate}
	\item asymptotically return to $\bm{x}_0$ \mayb{(asymptotically stable),}
	\item diverge from $\bm{x}_0$ \mayb{(unstable), or}
	\item remain perturned or oscillate about $\bm{x}_0$ with a constant amplitude \mayb{(marginally stable).}
\end{enumerate}	

A \keyword{phase portrait} is a parametric plot of state variable \keyword{trajectories}, with time implicit.
Phase portraits are exceptionally useful for understanding nonlinear systems, but they also give us a nice way to understand stability, as in \autoref{fig:phase_portrait_stability}.
\tags{}

\begin{figure}[tb]
\newcommand{\phasespacewidth}{.30\linewidth}
\centering
\begin{tabular}{ccc}
\includegraphics[width=\phasespacewidth]{figures/phase_space_stable.pdf} & 
\includegraphics[width=\phasespacewidth]{figures/phase_space_marginal.pdf} & 
\includegraphics[width=\phasespacewidth]{figures/phase_space_unstable.pdf} 
\end{tabular}
\caption{a phase-portrait demonstration of (left) asymptotic stability, (center) marginal stability, and (right) instability for a second-order system.}
\label{fig:phase_portrait_stability}
\end{figure}

These definitions of stability can be interpreted in terms of the free response of a system, as described, below.
\tags{}

\subsection{Stability defined by the free response} 
\tags{}
\label{sec:stabilityfree}


Using the concept of the free response (no inputs, just initial conditions), we define the following types of stability for LTI systems~\cite{Nise2015}.
\tags{}
\begin{enumerate}
	\item An LTI system is \keyword[asymptotic stability]{asymptotically stable} if the free response approaches an equilibrium state as time approaches infinity.
	\item An LTI system is \keyword[instability]{unstable} if the free response grows without bound as time approaches infinity.
	\item An LTI system is \keyword[marginal stability]{marginally stable} if the free response neither decays nor grows but remains constant or oscillates as time approaches infinity.
\end{enumerate}

These statements imply that the free response alone governs stability.
Recall that the free response $y_\text{fr}$ of a system with characteristic equation roots $\lambda_i$ with multiplicity $m_i$, for constants $C_i$, is
\tags{}
\maybeeq{
	\begin{align*}
		y_\text{fr}(t) = \sum_i C_i t^{m_i-1} e^{\lambda_i t}.
	\end{align*}
}

\noindent Each term will either decay to zero, remain constant, or increase without bound---depending on the sign of the \emph{real part} of the corresponding root of the characteristic equation $\Re{(\lambda_i)}$.
\tags{}

In other words, for an LTI system, the following statements hold.
\begin{enumerate}
	\item An LTI system is \emph{asymptotically stable} if, for all $\lambda_i$, $\Re{(\lambda_i)} < 0$.
	\item An LTI system is \emph{unstable} if, for any $\lambda_i$, $\Re{(\lambda_i)} > 0$.
	\item An LTI system is \emph{marginally stable} if, for all $\lambda_i$, $\Re{(\lambda_i)} \le 0$ and at least one $\Re{(\lambda_i)} = 0$.
\end{enumerate}

\subsection{Stability defined by the forced response}
\tags{}

An alternate formulation of the stability definitions above is called the \keyword[BIBO stability]{bounded-input bounded-output} (BIBO) definition of stability, and states the following~\cite{Nise2015}.
\tags{}
\begin{enumerate}
	\item A system is \keyword{BIBO stable} if every bounded input yields a bounded output.
	\item A system is \keyword{BIBO unstable} if any bounded input yields an unbounded output.
\end{enumerate}

In terms of BIBO stability, \keyword[BIBO marginal stability]{marginal stability}, then, means that a system has a bounded response to some inputs and an unbounded response to others. For instance, a second-order undamped system response to a sinusoidal input at the natural frequency is unbounded, whereas every other input yields a bounded output.
\tags{B, D}

Although we focus on the definitions of stability in terms of the free response, it is good to understand BIBO stability, as well.
\tags{}

\section{Vibration isolation table analysis}
\tags{}

\begingroup
\begin{wrapfigure}[10]{r}{0.35\textwidth}
  \centering
  \begin{tikzpicture}
		\coordinate (n1) at (0,0);
		\draw[<-,>=stealth',thick,violet] (n1) -- ++(0,-.75)
			node[left,black,midway,anchor=east] {$V_s$};
		\draw ($(n1)+(-.5,0)$)
			coordinate (n1l) -- ($(n1)+(.5,0)$)
			coordinate (n1r);
		\node[graphnode] at (n1) {};
		\draw[spring] (n1l) 
		  --node[left=7pt,anchor=east]{$k$} ++(0,2)
			coordinate (n2l) {};
		\draw[damper] (n1r) 
		  --node[left=7pt,anchor=east]{$B$} ++(0,2)
			coordinate (n2r) {};
		\node[draw,thick,minimum width=3cm,minimum height=.2cm,anchor=south] at ($(n1)+(0,2)$) {$m$};
	\end{tikzpicture}
  \caption{\label{fig:vibration_table_schematic} a vibration isolation table schematic with input velocity $V_s$.}
\end{wrapfigure}
In this example, we exercise many of the methods for modeling and analysis explored thus far.

Given the vibration isolation table model in \autoref{fig:vibration_table_schematic}---with $m = 320$ kg, $k = 16000$ N/m, and $B = 1200$ N--m/s---derive:
\begin{enumerate}
	\item a linear graph model,
	\item a state-space model,
	\item the equilibrium state $\bm{x}_0$ for the unit step input,
	\item a transfer function model,
	\item an input-output differential equation model with input $V_s$ and output $v_m$,
	\item a solution for $v_m(t)$ for a unit step input $V_s(t) = 1$ m/s for $t\ge 0$,
	\item the system's stability.
\end{enumerate}

\endgroup

\subsection{Linear graph and state-space models}
\tags{}

\begingroup
\begin{wrapfigure}[10]{r}{0.35\textwidth}
  \centering
	\begin{tikzpicture}[]
		\coordinate (g) at (0,0);
		\draw (g) pic {groundnode};
		\node[graphnode] (n1) at (-1.25,2) {};
		\node[graphnode] (n2) at (1.25,2) {};
		\draw[sourcebranch,color=mygreen,very thick] (n1) to[bend right] node[midway,below=7pt,left=7pt,anchor=north east] {$V_s$} (g);
		\draw[branch] (n1) to[bend left] node[above] {$k$} (n2);
		\draw[branch] (n1) to[bend right] node[below] {$B$} (n2);
		\draw[branch,color=mygreen,very thick] (n2) to[bend left] node[right=3pt] {$m$} (g);
	\end{tikzpicture}%
  \caption{\label{fig:vibration_table_linear_graph} linear graph of the isolation table.}
\end{wrapfigure}

The linear graph and normal tree are shown in \autoref{fig:vibration_table_linear_graph}.
The state variables are the velocity of the mass $v_m$ and the force through the spring $f_k$ and the system order is $n = 2$.
The input, state, and output vectors are
\tags{}
\begin{align}
	\bm{u} = \begin{bmatrix} V_s \end{bmatrix} \qquad
	\bm{x} = \begin{bmatrix} v_m \\ f_k \end{bmatrix} \qquad
	\bm{y} = \begin{bmatrix} v_m \end{bmatrix}.
\end{align}
The elemental equations are as follows.\\
\begin{tabular}{l|l}
$m$ & $\dot{v}_m = \dfrac{1}{m} f_m$ \\
$k$ & $\dot{f}_k = k v_k$ \\
$B$ & $f_B = B v_B$
\end{tabular}

\endgroup

\vspace{.25\baselineskip}

\noindent The continuity and compatibility equations are as follows.\\
\begin{tabular}[t]{r|l}
branch & continuity equation \\ \hline
$m$ & $f_m = f_k + f_B$
\end{tabular}
\qquad
\begin{tabular}[t]{r|l}
link & compatibility equation \\ \hline
$k$ & $v_k = V_s - v_m$ \\
$B$ & $v_B = V_s - v_m$ 
\end{tabular}

\noindent 

\noindent The state equation can be found by substituting the continuity and compatibility equations into the elemental equations, and eliminating $f_B$, to yield
\tags{}
\begin{subequations}
\begin{align}
	\dot{\bm{x}} &= 
	\begin{bmatrix}
		-B/m & 1/m \\
		-k & 0
	\end{bmatrix}
	\bm{x} + 
	\begin{bmatrix}
		B/m \\ k
	\end{bmatrix}
	\bm{u} \\
	\bm{y} &= 
	\begin{bmatrix}
		1 & 0
	\end{bmatrix}
	\bm{x}
	+
	\begin{bmatrix} 
		0 
	\end{bmatrix}
	\bm{u}.
\end{align}
\end{subequations}

\subsection{Equilibrium}
\tags{}

Let's check to see if $A$ is invertible by trying to compute its inverse:
\begin{align}
  A^{-1} &= 
	\begin{bmatrix}
		-B/m & 1/m \\
		-k & 0
	\end{bmatrix}^{-1} \\
	&= \frac{1}{k/m}
	\begin{bmatrix}
		0 & -1/m \\
		k & -B/m
	\end{bmatrix}
\end{align}
So it has an inverse, after all! Let's use this to compute the equilibrium state:
\begin{align}
	\bm{x}_0 &= -A^{-1}B\overline{\bm{u}} \\
	&= 
	\frac{-m}{k}
	\begin{bmatrix}
		0 & -1/m \\
		k & -B/m
	\end{bmatrix}
	\begin{bmatrix}
		B/m \\ k
	\end{bmatrix}
	\begin{bmatrix}
		1
	\end{bmatrix} \\
	&= 
	\frac{-m}{k}
	\begin{bmatrix}
		-k/m \\
		0
	\end{bmatrix}\\
	&= 
	\begin{bmatrix}
		1 \\
		0
	\end{bmatrix}
\end{align}

So the system is in equilibrium with $v_m = 1$ m/s and $f_k = 0$ N.
Since $v_m$ is also our output, we expect $1$ m/s to be our steady-state output value.
\tags{}

\subsection{Transfer function model}
\tags{}

The transfer function $H(s) = V_m(s)/V_s(s)$ will be used as a bridge to the input-output differential equation.
The transfer function can be found from the usual formula, from
\tags{}
\autoref{lec:bridge_state_space_to_io},
\begin{align}
	H(s) = C(sI - A)^{-1}B + D.
\end{align}
Let's first compute $(sI - A)^{-1}$:\footnote{See \cite[Sec.~A.4.3]{Rowell1997} for details on the matrix inverse.}
\begin{subequations}
\begin{align}
	(sI - A)^{-1} &= \left(
		\begin{bmatrix}
			s & 0 \\
			0 & s
		\end{bmatrix}
		-
		\begin{bmatrix}
			-B/m & 1/m \\
			-k & 0
		\end{bmatrix}
	\right)^{-1} \\
	&=
		\begin{bmatrix}
			s+B/m & -1/m \\
			k & s
		\end{bmatrix}^{-1} \\
	&=
		\frac{1}{(s+B/m)(s)-(-1/m)(k)}
		\begin{bmatrix}
			s & 1/m \\
			-k & s+B/m 
		\end{bmatrix} \\
	&=
		\frac{1}{s^2+(B/m) s + k/m}
		\begin{bmatrix}
			s & 1/m \\
			-k & s+B/m 
		\end{bmatrix}
\end{align}
\end{subequations}
Now we're ready to compute the entirety of $H$:
\begin{subequations}
\begin{align}
	H(s) &= 
		\frac{1}{s^2+(B/m) s + k/m}
		\begin{bmatrix}
			1 & 0
		\end{bmatrix}
		\begin{bmatrix}
			s & 1/m \\
			-k & s+B/m 
		\end{bmatrix}
		\begin{bmatrix}
			B/m \\ k
		\end{bmatrix}
		+
		\begin{bmatrix}
			0
		\end{bmatrix} \\
	&= 
		\frac{1}{s^2+(B/m) s + k/m}
		\begin{bmatrix}
			s & 1/m
		\end{bmatrix}
		\begin{bmatrix}
			B/m \\ k
		\end{bmatrix} \\
	&= 
		\frac{(B/m) s + k/m}{s^2+(B/m) s + k/m}.
\end{align}
\end{subequations}

\subsection{Input-output differential equation}
\tags{}

The input-output differential equation can be found from the reverse of the procedure in \autoref{lec:bridge_state_space_to_io}.
Beginning from the transfer function,
\tags{}
\begin{subequations}
\begin{align}
	\frac{V_m}{V_s} &= \frac{(B/m) s + k/m}{s^2+(B/m) s + k/m} \Rightarrow \\
	\left(s^2+(B/m) s + k/m\right) V_m &= 
	\left((B/m) s + k/m\right) V_s \Rightarrow \\
	\ddot{v}_m + (B/m) \dot{v}_m + (k/m) v_m &=
	(B/m) \dot{V}_s + (k/m) V_s. \label{eq:vibration_table_io_differential_equation}
\end{align}
\end{subequations}

\subsection{Step response}
\tags

The step response is found using superposition and the derivative property of LTI systems.
The forcing function $f(t) = (B/m) \dot{V}_s + (k/m) V_s$ is composed of two terms, one of which has a derivative of the input $V_s$.
Rather than attempting to solve the entire problem at once, we choose to find the response for a forcing function $f(t) = 1$ (for $t\ge 0$)---that is, the unit step response---and use superposition and the derivative property of LTI systems to calculate the composite response.
\tags{}

\subsubsection{Unit step response}
\tags{}

The characteristic equation of \autoref{eq:vibration_table_io_differential_equation} is
\begin{subequations}
\begin{align}
	\lambda^2 + (B/m) \lambda + k/m &= 0 \Rightarrow \\
	&= -\frac{B}{2 m} \pm \frac{\sqrt{B^2 - 4 m k}}{2 m} \Rightarrow \\
	\lambda_{1,2}&= -1.875 \pm j 6.818.
\end{align}
\end{subequations}

The roots are complex, so the system will have a damped sinusoidal step response.
Let $\sigma = -1.875$ and $\omega = 6.818$ such that $\lambda_{1,2} = \sigma \pm j\omega$.
The homogeneous solution is
\tags{}
\begin{align}
	v_{m_h}(t) = C_1 e^{\lambda_1 t} + C_2 e^{\lambda_2 t}.
\end{align}
In this form, $C_1$ and $C_2$ are complex.
It is somewhat easier to deal with 
\begin{subequations}
\begin{align}
	v_{m_h}(t) &= C_1 e^{\sigma t} e^{j\omega t} + C_2 e^{\sigma t} e^{-j\omega t} \\
	&= e^{\sigma t}
		\left(
			C_1 \cos\omega t + j C_1 \sin\omega t +
			C_2 \cos\omega t - j C_2 \sin\omega t
		\right)\\
	&= e^{\sigma t}
		\left(
			(C_1+C_2) \cos\omega t + j (C_1-C_2) \sin\omega t
		\right).
\end{align}
\end{subequations}
Let $C_3 = C_1+C_2$ and $C_4 = j (C_1-C_2)$ such that
\begin{align}
	v_{m_h}(t) = e^{\sigma t}
		\left(
			C_3 \cos\omega t + C_4 \sin\omega t
		\right).
\end{align}

This is a decaying (because $\sigma<0$) sinusoid.
A nice aspect of this new form is that $C_3$ and $C_4$ are real.
\tags{}

Now, the particular solution can be found by assuming a solution of the form $v_{m_p}(t) = K$ for $t\ge 0$.
Substituting this into \autoref{eq:vibration_table_io_differential_equation} (with forcing $f(t) = 1$, we attempt to find a solution for $K$ (that is, \emph{determine} it):
\tags{}
\begin{align}
	(k/m) K = 1 \Rightarrow K = m/k.
\end{align}	
Therefore, $v_{m_p}(t) = m/k$ is a solution, and therefore the general solution is
\begin{subequations}
\begin{align}
	v_{m_\text{step}}(t) &= v_{m_h}(t) + v_{m_p}(t) \\
	&= e^{\sigma t}
		\left(
			C_3 \cos\omega t + C_4 \sin\omega t
		\right)
		+
		m/k.
\end{align}
\end{subequations}

This leaves the specific solution, to be found applying the initial conditions (assumed to be zero).
Before we do so, however, we need the time-derivative of the $v_{m_\text{step}}$:
\tags{}
\begin{align}
	\dot{v}_{m_\text{step}}(t) &= e^{\sigma t} 
	\left(
		(C_3\sigma+C_4\omega) \cos(\omega t) +
		(C_4\sigma-C_3\omega) \sin(\omega t)
	\right).
\end{align}
Now, applying the initial conditions,
\begin{subequations}
\begin{align}
	v_{m_\text{step}}(0) &= 0 \Rightarrow \\
		C_3 &= -m/k \\
	\dot{v}_{m_\text{step}}(0) &= 0 \Rightarrow
		0 = C_3\sigma + C_4\omega \Rightarrow \\
		C_4 &= \frac{\sigma}{\omega}\cdot\frac{m}{k}.
\end{align}
\end{subequations}

It's good form to re-write this as a single sinusoid:
\begin{subequations}
\begin{align}
	v_{m_\text{step}}(t) &= v_{m_h}(t) + v_{m_p}(t) \\
	&= A_1 e^{\sigma t} \cos(\omega t + \psi_1)
		+
		m/k
\end{align}
\end{subequations}
where we have used \autoref{eq:two_to_one} to find
\begin{subequations}
\begin{align}
	A_1 &= \sqrt{C_3^2 + C_4^2} \\
	\psi_1 &= -\arctan(C_4/C_3).
\end{align}
\end{subequations}

\subsubsection{Superposition and the derivative property}


Recall that the actual forcing function is a linear combination of the input and its time-derivative.
Therefore, it is expedient to re-write the time-derivative of the unit step response:
\tags{}
\begin{subequations}
\begin{align}
	\dot{v}_{m_\text{step}}(t) &= A_1 e^{\sigma t} 
	\left(
		\sigma \cos(\omega t + \psi_1) -
		\omega \sin(\omega t + \psi_1)
	\right) \\
	&= A_1 A_2 e^{\sigma t}\cos(\omega t + \psi_1 + \psi_2)
\end{align}
where
\begin{align}
	A_2 &= \sqrt{\sigma^2 + \omega^2} \\
	\psi_2 &= -\arctan(-\omega/\sigma).
\end{align}
\end{subequations}
Finally, applying superposition and the derivative rule of LTI systems,
\begin{subequations}
	\begin{align}
		v_m(t) &= (B/m) \dot{v}_{m_\text{step}}(t) + (k/m) v_{m_\text{step}} \\
		&= 
		\frac{B}{m} A_1 A_2 
		e^{\sigma t}\cos(\omega t + \psi_1 + \psi_2) 
		+
		\frac{k}{m} A_1 e^{\sigma t} \cos(\omega t + \psi_1)
		+
		1.
	\end{align}
\end{subequations}
This is the solution!

It's worth plotting the response.
\input{source/vibration_table_example}

Note that the steady-state output value agrees with that predicted by the equilibrium analysis, above.
\tags{}

\subsection{Stability}
\tags{}

We have learned what we need in order to analyze the system's stability.
The roots of the characteristic equation were $\lambda_{1,2} = -1.875 \pm j 6.818$, which clearly \emph{all} have negative real parts, and therefore the system is \emph{asymptotically stable}.
\tags{}

\end{document}